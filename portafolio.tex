% Formato de portafolio
% Documento adaptado de https://github.com/PlasmaPhysicsUdeC/FondecyTeX por Roberto Navarro <roberto.navarro@udec.cl>
% Este documento es modular, es decir, esta separado en varios
% archivos para facilitar su uso. Ver uso del paquete `subfiles`:
% https://www.overleaf.com/learn/latex/Multi-file_LaTeX_projects

\documentclass[twoside]{report}
\usepackage[table]{xcolor}
\usepackage[paper=letterpaper,left=3cm,right=3cm,top=3cm,bottom=2cm,includefoot]{geometry}
\usepackage{fancyhdr}
\usepackage{tabularx}
\usepackage{multirow}

\usepackage{hhline}
\usepackage{amsmath}
\usepackage{enumitem}
\setenumerate{itemsep=-3pt,topsep=3pt}
\setdescription{itemsep=-3pt,topsep=3pt,leftmargin=!,labelwidth=4.5cm}
\usepackage{pgfgantt}
\usepackage{graphicx}
\graphicspath{{./img/} {./tex/img/} {../img/} }

\usepackage[spanish]{babel}
\usepackage[utf8]{inputenc}
\usepackage[T1]{fontenc}
\spanishdecimal{.}

% bibliografia: descomente estas dos lineas para usar estilo numerico, e.g. [1].
\usepackage[square,numbers,sort&compress]{natbib}
\bibliographystyle{apsrev4-1}




\usepackage{listings}
\usepackage{xcolor}

% Para incluir codigos python.
% Necesita opcion -shell-escape para compilar
\usepackage[cachedir=/tmp/minted-caches]{minted}
% Estilo para todo tipo de codigo
\setminted{breaklines,
	xleftmargin=\parindent,
	numbersep=5pt,
	bgcolor=lightgray!30,
	fontsize=\small
}
\usemintedstyle{manni}
% Estilo solo para python
\setminted[python]{linenos,
	texcomments,
	mathescape,
	texcomments,
}

% subfiles permite que el documento sea modular
\usepackage{subfiles}

%%% Comente/Descomente las siguientes lineas para cambiar la fuente del texto
% \usepackage{DejaVuSans}
% \renewcommand*\familydefault{\sfdefault}
% \usepackage{sansmath}
% \sansmath

% \numberwithin{equation}{chapter}

\pagestyle{fancy}
\renewcommand{\footrulewidth}{0.4pt}
\renewcommand{\headrulewidth}{0.4pt}
\fancyfoot{}
\fancyfoot[RE,RO]{\thepage}
\fancyfoot[LO,LE]{\textcolor[RGB]{127,127,127}{Portafolio - Física Computacional II (2024)}}

\fancyhead[LE,RO]{}
\definecolor{tcc}{RGB}{217,217,217} % Table cell color

\renewcommand\tabularxcolumn[1]{m{#1}}
\setlength{\arrayrulewidth}{0.5pt}
\renewcommand{\arraystretch}{2}

\usepackage{listings} 
\definecolor{bg}{rgb}{0.95,0.95,0.95} 
\definecolor{keywords}{RGB}{0,0,255} 
\definecolor{comments}{RGB}{0,128,0} 
\definecolor{strings}{RGB}{163,21,21} 
\lstset{ backgroundcolor=\color{bg}, basicstyle=\small, breaklines=true, commentstyle=\color{comments}\itshape, keywordstyle=\color{keywords}\bfseries, stringstyle=\color{strings}, numbers=left, numbersep=5pt, frame=single, rulecolor=\color{black}, xleftmargin=\parindent, showstringspaces=false, language=Python }


% \renewcommand{\thesection}{\alph{section})}
% \renewcommand{\thesubsection}{\alph{section}.\arabic{subsection}}
\usepackage[colorlinks]{hyperref}
\hypersetup{
    citecolor=cyan!90!black,
    urlcolor=cyan!80!purple
}

\begin{document}

% ASEGÚRESE DE COLOCAR SUS DATOS EN LA PORTADA
\subfile{tex/portada}


\tableofcontents

\nocite{*}
%%% con \subfile se incluyen archivos externos
\subfile{tex/introduccion}
\subfile{tex/presentacion-estudiante}

\part*{Evidencias de aprendizaje}  % cuerpo del portafolio
\addcontentsline{toc}{part}{Evidencias de aprendizaje}
\markboth{Evidencias de aprendizaje}{Evidencias de aprendizaje}
%controles
\subfile{tex/control_1}
\subfile{tex/control_2}
\subfile{tex/control_3}
\subfile{tex/control_04}
\subfile{tex/ej6_NR}
%\subfile{tex/control_6}
%guia 1
\subfile{tex/numeros-de-catalan}
\subfile{tex/secuencia_Lucas}
\subfile{tex/exponencialTaylor_guia1}
%guia 2
\subfile{tex/g2_pesos}
\subfile{tex/g2_ej8}
\subfile{tex/guia_2_ej_9}
%\subfile{}
%guia 3
%\subfile{tex/pesos_integracion}
\subfile{tex/funcion_gamma}
\subfile{tex/guia_3_matriz}
%guia 4
\subfile{tex/g4_ej5}
\subfile{tex/g4_pts}
\subfile{tex/g4_ej11}
%guia 5
\subfile{tex/g5_ej8}
\subfile{tex/g5_legendre}
\subfile{tex/g5_ej4}
%guia 6
%\subfile{}
%\subfile{}
%\subfile{}
% Conclusiones y referencias
\subfile{tex/conclusion-global}
\subfile{tex/agrdecimientos}

%%%%%%%%%%%%%%%%%%%%%%%%%%%%%%%%%%%%%%%%%%%%%%%%%%%%%%%%%%%%%%%%%%%%%%
% Puede usar el capítulo de apéndice para agregar sus códigos completos si lo desea
\appendix
\subfile{tex/gravitacion_edo}
%\subfile{tex/...}

% lista de referencias guardadas en referencias.bib
\bibliography{referencias}

\end{document}