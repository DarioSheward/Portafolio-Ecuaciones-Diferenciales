\documentclass[../portafolio.tex]{subfiles}

\begin{document}

\chapter{Polinomios de Hermite}
\label{g5_ej4}
\hfill \textbf{Fecha de la actividad:} 27 de noviembre de 2024

\medskip

En este cápitulo se estiman numéricamente los polinomios de Hermite, mediante su relación de recurrencia y esquemas de derivación centrada, como también se determinan los ceros de cada una de estas funciones a través del método de la secante. Luego, se revisan los factores de error y se gráfican los resultados.

\section*{Objetivos}
\begin{itemize}
\item Determinar los primeros 5 polinomios de Hermite mediante el uso de la relación de recurrencia y derivadas numéricas centradas.
\item Determinar los ceros de los primeros 5 polinomios de Hermite en el intervalo $-3\leq x \leq 3$.
\end{itemize}
\medskip

Los polinomios de Hermite son una familia de polinomios ortogonales de frecuente aparición en la física, particularmente en problemas relacionados al oscilador armónico cuántico. Definiremos a éstos con la relación de recurrencia 
\begin{equation}
H_{n+1}(x)=2x H_n (x) - H'_n(x),
\end{equation}
donde, $H'_n(x)$ es la primera derivada de $H_n(x)$. Si $H_0 (x)=1$, se nos pide usar la relación de recurrencia y las derivadas numéricas para realizar lo propuesto en los objetivos.
\section{Los primeros 5 polinomios de Hermite}
A todo este ejercicio se le dará un enfoque numérico, tanto determinando los polinomios, como determinando los cero de estos, por muy fácil que fuera usando funciones integradas en \texttt{numpy}. Antes de empezar con la explicación del código, destacar como el enfoque de el script fueron los conjuntos de datos y como trabajarlos.\\
El esquema de derivadas centradas utilizado fue:
\begin{equation}
f'(x_n)=\frac{f(x_{n+1})-f(x_{n-1})}{x_{n+1}-x_{n-1}}.
\end{equation}
\begin{minted}{python}
#Se define un esquema de derivadas centradas para arreglo de numéros, junto al arreglo de su función evaluada y junto al orden del esquema.
def derivadacentrada(fx,x,o): #'fx' sea un arreglo con las x's ya evaluadas en la función.
    x_1=x[o:-o]               #'x' sea el conjunto de x's a evaluar.
    l=o*2   #'o' sea la cantidad de datos entre el x y donde se evalúa para la derivada. Se duplica para trabajar desde solo índice desplazado a un lado.
    n=len(x) -l
    df=np.zeros(n)            #Se prepara un arreglo para guardar el valor de las derivadas
    for i in range(n):        #Se inicia la revisión de cada índice de los arreglos.
        dx=np.abs(x[i+l]-x[i]) #Se define la distancia entre los puntos a evaluar.
        if dx!=0:
            df[i]=(fx[i+l]-fx[i])/(dx)  #Se considera el doble de la distancia 'o', para asi calcular 'df[i+o]'
        else:
            df[i]=np.nan
    return x_1, df #La función devuelve el conjunto de x cuyas derivadas se calcularon, junto a las últimas.
\end{minted}

\begin{minted}{python}
#Se define el proceso de estimación numérica de los polinomios.
def Hermite(n,x,o=1):
    H=[]    #se definen las listas donde se guardarán los arreglos de el dominio, la función evaluada y la derivada evaluada. (evaluadas/numéricas).
    dH=[]
    xs=[]
    xs.append(x)    #se establece el dominio de H 0(x).
    H.append(np.ones(len(x)))   #se establece H 0(x).
    fx=lambda x: 2*x    #Se define como función para poder cambiar sus dimensiones según las derivadas centradas acorten el dominio.
    for k in range(n+1):    #Ciclo donde se establece la derivada numérica de H n(x) y el nuevo dominio.
        xd , dh=derivadacentrada(H[k],xs[k],o)  
        dH.append(dh)       #Luego cada arreglo de información se guarda en las listas.
        xs.append(xd)
        H.append(fx(xd)*H[k][1:-1]-dH[k])
    return xs ,H, dH        #Se devuelven las listas con los arreglos de los dominios y estimaciones numéricas de las funciones.
\end{minted}
El método de estimación de ceros utilizado fue el método de la secante, debido a las condiciones que generamos al evaluar el dominio y conservar la información como arreglo. En función de los términos se representa el método como:
\begin{equation}
x_{n+1}=x_n - f(x_n)\frac{x_{n+1}-x_{n-1}}{f(x_{n+1})-f(x_{n-1})}.
\end{equation}
\begin{minted}{python}
def secante(x,f,df): #definimos el método de la secante para aplicarlo sobre un polinomio a la vez. Se necesitará de su dominio, el polinomio evaluado, y su derivada numérica.
    x=x[1:-1]   #Se adapta el tamaño del dominio a el de la derivada, debido a la pérdida de datos extremos durante la derivación numérica.
    tol=1e-2    #Se establece una tolerancia a partir de la cual el ciclo no necesita llegar al cero sino que solo aproximarse lo suficiente.
    ceros=[]    #Una lista donde guardar y devolver los ceros.
    itmax=1000  #Máximo de iteraciones.
    seeds=np.linspace(-3,3,10)  #Se establece un arreglo con 10 semillas en el intervalo deseado.
    for i in seeds:         #Se selecciona una semilla a la vez.
        n=(np.abs(x-i)).argmin()    #Se busca el índice cuyo x sea el más cercano a la semilla en todo el arreglo.
        it=0    #Se empieza el conteo de iteraciones.
        if np.abs(x[n])<tol:    #Si la semilla es una cercana a 0, ésta se guarda en la lista ceros.
            ceros.append(x[n])
        elif np.abs(df[n])<tol: #Si la derivada en el índice seleccionado es cero, se avanza a la siguiente semilla.
            break
        else:
            while np.abs(f[n])>tol and it<itmax:    #De estar x[n] lejano a cero, entonces se aplica el método de la secante.
                x2=x[n]- (f[n]/df[n])   #Primero, se estima el siguiente punto sugerido por el método.
                n=(np.abs(x-x2)).argmin()   #Luego, se busca el índice del x más cercano al nuevo x sugerido.
                it+=1   #Se cuenta el número de iteraciones.
            ceros.append(x[n])  #Si al evaluar al nuevo x[n], éste es menor a la tolerancia, entonces se registra el punto x en el arreglo.
    return ceros    #Se retornan los x de los ceros la función.
\end{minted}

\begin{minted}{python}
x=np.linspace(-3.1,3.1,1000)    #Se establece un dominio.
HHH=Hermite(4,x)    #Se calculan los primeros 5 polinomios, sus dominios y sus derivadas.
i=1 #Se inicia el ciclo desde 1 ya que la función de Hermite define su termino n=0 al comenzar la función. Además se hace para evitar dividir por cero al aplicar el método de la secante.
for i in range(5):
    ceros= secante(HHH[0][i], HHH[1][i], HHH[2][i])  #Se ingresan los arreglos de un polinomio por ciclo y guarda los ceros de este polinomio.
    plt.scatter(ceros,np.zeros(len(ceros)))     #Se grafican los ceros junto a las curvas de los polinomios.
    plt.plot(HHH[0][i],HHH[1][i], label=f'$H_{i}(x)$')
\end{minted}
El script se dedica a realizar estimaciones numéricas a partir de conjuntos de datos y guardarlos en listas, para luego, nuevamete trabajar con el conjunto y volver a empaquetarlo para que sea acompañado de sus similes de otras funciones. Así finalmente con un ciclo se pasa por los índices de las listas para sacar los conjuntos de datos y graficarlos. 

El resultado de este código es la figura \ref{g5_ej4_graf}

\begin{figure}
\centering
\includegraphics[scale=0.75]{../img/g5_hermite.png}
\caption{Primeros 5 polinomios de Hermite definidos numéricamnte, junto a sus ceros, definidos mediante una variación del método de la secante.} \label{g5_ej4_graf}
\end{figure}
\section{Análisis de resultados}
%Que pasa si el cero esta afuera del intervalo
En general el proceso realizado presenta errores únicamente debido a la presencia de la tolerancia, por ello los erreos son del orden de $10^{-2}$ o menores. Según la metodología utilizada en el script, este error puede variar dependiendo de cuanto espacio haya entre cada x, el número máximo de iteraciones y de la tolerancia seleccionada.
\section*{Conclusiones}
Este ejercicio se afrontó con métodos puramente numéricos, dando espacio a errores estimables, principalmente al encontrar ceros. La obtención numérica de los polinomios resultó útil, sin embargo, queda pendiente determinar los coeficientes de éstos. La recursividad de la relación permitio el registro de los cálculos, para evadir la reiteración de éstos, lo que resultó muy útil en cuanto a eficiencia computacional se refiere. 
Teniendo eso en consideración, se cumplieron los objetivos propuestos, aunque quisiera volver a revisar el problema para determinar los coeficientes de los polinomios.


\end{document}
