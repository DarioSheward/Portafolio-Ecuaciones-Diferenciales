\documentclass[../portafolio.tex]{subfiles}

\begin{document}

\chapter{Función Gamma}
\label{g3_ej10}
\hfill \textbf{Fecha de la actividad:} 15 de noviembre de 2024

\medskip

Se presentan métodos de estimación numérica para integrales, en particular, reglas que acomodan a la peculiar función Gamma. Una función de $x$ donde se evalúa una integral en función de $t$ de límites de integración semi-infinitos. Para ello se utilizan las reglas de Simpson y de Gauss-Laguerre, para luego compararlas gráficamente.

\section*{Objetivos}
\begin{itemize}
\item Encontrar la función Gamma usando una regla de Simpson compuesta.
\item Aplicar la regla de Gauss-Laguerre para calcular la función.
\item Comparar los métodos de cálculo.
\end{itemize}
Como fue visto en el capítulo \ref{g1_ej2}, la función Gamma es la función factorial. Ahora se calculará para $x>0$ y se compararán varios métodos estimaciones numéricas de integrales como lo es:
\begin{equation}\label{g3_ej10:eq:f_gamma}
\Gamma(x)=\int_{0}^{\infty}t^{x-1}e^{-t}\quad dt, \quad x>0.
\end{equation}
\section{Reglas de Simpson compuestas}
Esta primera regla utiliza 3 puntos de una curva para aproximar el área debajo de ésta. Para esto trabaja con la interpolación de polinomio de grado dos, es decir parábolas, para ajustarles a la función. Donde para un sólo intervalo $[a,b]$ se puede usar:

\begin{equation}\label{g3_ej10:eq:simpgeneral}
\int_{a}^{b} f(x)\quad dx =\frac{b-a}{6}[f(a)+4f(m)+f(b)]-\frac{(b-a)^5}{90}f^{(4)}(\eta)
\end{equation}

Quede la demostración de la interpolación para otra ocasión. \\
Ahora, se separa el intervalo en subintervalos iguales, $h=(b-a)/n$, donde $n$ debe ser par, aplicando la regla a cada uno de estos. Donde cada uno queda como se muestra en \eqref{g3_ej10:eq:1/3simpson}.
\begin{equation}\label{g3_ej10:eq:1/3simpson}
\int_{x_{i-1}}^{x_{i+1}} f(x)\quad dx \approx\frac{h}{6}[f(x_{i-1})+4f(x_i)+f(x_{i+1})]
\end{equation}
Sumando todos los subintervalos, resulta:
\begin{equation} \label{g3_ej10:eq:simpson}
\int_{a}^{b} f(x)\quad dx \approx \frac{h}{3} \left[ f(a)+2\sum_{k=1}^{\frac{n}{2}-1}f(x_2k)+4\sum_{k=1}^{\frac{n}{2}}f(x_{2k-1}+f(b)\right]
\end{equation}

Es evidente que la regla utiliza límites de integración superiores. Dado que \eqref{g3_ej10:eq:f_gamma} tiende a infinito por tal límite, tendremos que acotarlo. Para esto consideraremos una función de comportamiento similar que nos permita reconocer que intervalo de $t$ es lo suficientemente grande para capturar la mayor parte del área bajo la curva sin desperdiciar recursos computacionales en cálculos innecesarios. Para eso usaremos la función \eqref{g3_ej10:eq:xd}.
\begin{equation}\label{g3_ej10:eq:xd}
f(x,t)=t^{x-t}e^{-t}
\end{equation}

A continuación se programa la computadora para detener su avance en \texttt{t} en cuanto la función evaluada en un punto \texttt{x} ya sea menor a la tolerancia elegida.

\begin{minted}{python}
def convergencia(x=0.001,tol=1e-36):	#Definimos una función convergencia que nos indicará hasta que valor de t podemos evaluar efectiva, evitando sumar valores muy cercanos a cero. En este caso la tolerancia es la capacidad mínima del sistema de precisión simple.
    t=0.1
    while True:    
        f=np.abs(t**(x-t)*np.exp((-1)*t))
    
        t+=1
        if f<tol:
            break
    return t
tmax=convergencia(x)
\end{minted}

Con un límite de integración superior ahora podemos calcular. Se escribe el script para calcular la expresión \eqref{g3_ej10:eq:simpson}. Donde se preparan los \texttt{n} subintervalos de tamaño \texttt{h}, y luego se registran las sumas por cada \texttt{x} evaluada.

\begin{minted}{python}
def SimpComp(x , n , f , tmax , tmin=1e-6 ):    #La regla de Simpson necesita de la cantidad de intervalos, el tiempo máximo, un dominio y una función donde se haya evaluado el dominio (estos ultimos del tamaño n+1).
    h=(tmax-tmin)/n
    if n mod 2 == 1: #n debe ser par, por lo que si no lo es se le suma uno.
        n += 1
    puntos=np.empty(len(x))
    for i in range(len(x)):		#Luego se realizan la suma de cada subintervalo con sus imágenes, con sus respectiva ponderación.
        sum=f(x[i],tmin)+f(x[i],tmax)
        for r in range(1, n, 2):
            sum+=4*(f(x[i],tmin + h*r))
        for r in range(2, n, 2):
            sum+=2*(f(x[i],tmin + h*r))
        puntos[i]=sum*(h/3)
    return puntos	#Se regresa el valor de la integral evaluada.
\end{minted}

\section{Regla de Gauss-Laguerre}
Ahora usaremos una cuadratura gaussiana \citep{Gamma0} para calcular la función. Debido a las características de \eqref{g3_ej10:eq:f_gamma} usaremos la regla de Gauss-Laguerre, específicamente por su factor exponencial y por su intervalo de integración semi-infinito \eqref{g3_ej10:eq:gau-lag}.

\begin{equation}\label{g3_ej10:eq:gau-lag}
\int_0^\infty e^{-x} f(x) \quad dx \approx \sum_{i=1}^n w_i f(x_i)
\end{equation}

Este método requiere de pesos que se asignan a ciertos puntos determinados a través de los polinomios de Laguerre soluciones a la ecuación diferencial de Laguerre y son ortogonales con respecto al peso $( e^{-x} )$.

Su implementación en Python requirió hacer una función que recibiera la cantidad de nodos a considerar, los \texttt{x} a evaluar, y la función preparada para la regla, en este caso $f(x)=t^{x-1}$. Para determinar los pesos y los nodos  se utiliza las funciones de \texttt{numpy} del subpaquete \texttt{polynomial}, notese que \texttt{laggauss(n)} \citep{numpy_laggauss} regresa un arrreglo de forma \texttt{(n,2)}.

\begin{minted}{python}
gamma_gauss=lambda x,t: t**(x-1)		# Sea esta la función a la que se aplicara la regla de Gauss-Laguerre.

def Gauss(f, n, x):    #Se define la regla de Gauss-Laguerre diseñada para int(e**-x * f(x)).
    nodos,pesos=laggauss(n)	#Se definen los nodos y los pesos a utilizar a partir de la función laggauss.
    sum=np.zeros(len(x))		#Se define un arreglo donde guardar los valores de la integral para cada x evaluado.
    for i in range(len(x)):	#Se determina la integral para cada x en el dominio.
        for n in range(len(puntos)):
            sum[i]+=(pesos[n]*f(x[i],nodos[n]))
    return sum	#Se retorna el arreglo de los valores de la función Gamma con respecto ax.
\end{minted}

\section{Comparación entre los métodos}
Durante el desarrollo del problema se realizaron múltiples gráficos, en su mayoría, el eje y en escala logarítmica.

La primera figura \ref{g3_ej10:fig:gam0} muestra los resultados de cada método, además de incluir la función Gamma definida en \texttt{scipy}. En este se presenta $0<x\leq 10$.

\begin{figure}
\centering
\includegraphics[scale=0.75]{../img/funcion_gamma_01.png} 
\caption{Gráfico comparando los distintos métodos de cálculo y la función.}\label{g3_ej10:fig:gam0}
\end{figure}

En la siguiente figura \ref{g3_ej10:fig:gam1}, se presentan los números enteros estimados con cada regla y comparados con la función factorial. Se evaluaron solo los números enteros que $1\leq x\leq 10$.

\begin{figure}
\centering
\includegraphics[scale=0.75]{../img/funcion_gamma_02.png} 
\caption{Gráfico comparando los distintos métodos de cálculo y la función.}\label{g3_ej10:fig:gam1}
\end{figure}

\section{Análisis de Resultados}

En cuanto a la figura \ref{g3_ej10:fig:gam0} es claro como las estimaciones presentan problemas con $x<1$, se exploraron las capacidades del script, tanto en la determinación del $t_{max}$, como en la cantidad de nodos, en ambas reglas, sin signos de mejora en las estimaciones.

Recogiendo el tema de los nodos, la regla de Simpson se ve beneficiada por el aumento en el número de nodos. Mientras la regla de Gauss-Lagurre es perjudicada a mayor número de nodos. Esto se puede explicar por como la primera funciona a través de la relación entre tres puntos seleccionados. Por el otro lado, la cuadratura gaussiana, asigna pesos y valores a partir de un algoritmo particular. Quede pendiente el análisis del polinomio de Laguerre.

Es evidente la superioridad en precisión y exactitud de la regla de Gauss-Laguerre al observar la figura \ref{g3_ej10:fig:gam2}. En general la función presenta un crecimiento rápido, por parte de los errores absolutos de las estimaciones crecen junto a la función, sin embargo, los errores relativos mantienen cierta constancia y permite tener una idea, al seguir con otros valores de x, de un error al considerar el método. Esto es aplicable a ambos métodos.

\begin{figure}
\centering
\includegraphics[scale=0.75]{../img/funcion_gamma_03.png} 
\caption{Gráfico comparando los distintos métodos de cálculo y la función.}\label{g3_ej10:fig:gam2}
\end{figure}
\section*{Conclusiones}
La función Gamma presenta un tipo particular de integral que necesitó de un método para determinar la convergencia de la función integrada a partir de cada elemento del dominio evaluado. Al menos eso para la regla de Simpson compuesta. Por otro lado, para resolver la integral impropia también se utilizó la regla de Gauss-Laguerre, cuyos resultados fueron muy satisfactorios. A la hora de compararlos fue notorio que una regla diseñada para este tipo de funciones impropias con factores exponenciales se logra una mayor precisión. Al reconocer lo último, el ejercicio muestra su función expositiva sobre los métodos utilizados.
\end{document}