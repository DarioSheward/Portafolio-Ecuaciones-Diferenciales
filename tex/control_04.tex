\documentclass[../portafolio.tex]{subfiles}

\begin{document}

\chapter{Control Ecuaciones Diferenciales Ordinarias}
\label{g0_c4}
\hfill \textbf{Fecha de la actividad:} 12 de noviembre de 2024

\medskip

En este capítulo se resuelve numéricamente un sistema de ecuaciones aplicando la ley de gravitación universal\footnote{En el apéndice \ref{g4_ej1} se encuentra un ejercicio similar sobre el problema de Kepler.} para dos masas, en dos dimensiones. Para esto se normalizan las ecuaciones y se dejan expresadas en términos adimensionales. Luego los resultados son graficados con Python.

\section*{Objetivos}
\begin{itemize}
\item Normalizar las ecuaciones de fuerza gravitatoria.
\item Reescribir el sistema de ecuaciones para utilizar el método del salto de la rana.
\item Escribir un script de Python resolviendo las ecuaciones numéricamente.
\item Graficar las trayectorias de las masas.
\end{itemize}

\section{Normalización de la ecuación}
Buscamos mostrar como el sistema de ecuación es \eqref{g0_c4:eq:gog} es normalizable con $\tau=t\sqrt{Gm_1/R_0^3}$, $\mu=\frac{m_2}{m_1}$ y $\vec{\xi}_i=\frac{\vec{r}_i}{R_0}$.
\begin{align} \label{g0_c4:eq:gog}
m_1 \frac{d^2 \vec{r}_1}{dt^2} = - \frac{G m_1 m_2 }{|\vec{r}_2 - \vec{r}_1|^3}(\vec{r}_1 - \vec{r}_2) \quad \wedge \quad
m_2 \frac{d^2 \vec{r}_2}{dt^2} = - \frac{G m_1 m_2 }{|\vec{r}_2 - \vec{r}_1|^3}(\vec{r}_2 - \vec{r}_1)
 \end{align}

Para esto se utiliza la regla de la cadena, sin embargo, en esta ocasión un abuso de notación. Reemplazamos en la primera ecuación de \eqref{g0_c4:eq:gog}.

\begin{align}
m_1\vec{\xi}_1\frac{d^2 \vec{r}_1}{d(\frac{\tau}{\sqrt{Gm_1/R_0^3}})^2}=&-\frac{Gm_1m_2}{|\vec{\xi}_2 - \vec{\xi}_1|^3 R_0^3}(\vec{\xi}_1 - \vec{\xi}_2)R_0\\
\frac{Gm_1^2}{R_0^2}\frac{d^2 \vec{\xi}_1}{d\tau^2}=&-\frac{Gm_1m_2}{|\vec{\xi}_2 - \vec{\xi}_1|^3 R_0^2}(\vec{\xi}_1 - \vec{\xi}_2)\\
\frac{d^2 \vec{\xi}_1}{d\tau^2}=&-\frac{\mu}{|\vec{\xi}_2 - \vec{\xi}_1|^3 }(\vec{\xi}_1 - \vec{\xi}_2)
\end{align}
Se hace un proceso similar en la segunda ecuación de \eqref{g0_c4:eq:gog}.
\begin{align}
m_2\vec{\xi}_2\frac{d^2 \vec{r}_2}{d(\frac{\tau}{\sqrt{Gm_1/R_0^3}})^2}=&-\frac{Gm_1m_2}{|\vec{\xi}_2 - \vec{\xi}_1|^3 R_0^3}(\vec{\xi}_2 - \vec{\xi}_1)R_0\\
\frac{Gm_1m_2}{R_0^2}\frac{d^2 \vec{\xi}_2}{d\tau^2}=&\frac{Gm_1m_2}{|\vec{\xi}_2 - \vec{\xi}_1|^3 R_0^2}(\vec{\xi}_1 - \vec{\xi}_2)\\
\frac{d^2 \vec{\xi}_2}{d\tau^2}=&\frac{1}{|\vec{\xi}_2 - \vec{\xi}_1|^3}(\vec{\xi}_1 - \vec{\xi}_2)
\end{align}

Resultando en un nuevo sistema de ecuaciones donde cada unos de sus términos son adimensionales.
\begin{equation}\label{g0_c4:eq.nuv}
\frac{d^2 \vec{\xi}_1}{d\tau^2}=-\frac{\mu}{|\vec{\xi}_2 - \vec{\xi}_1|^3 }(\vec{\xi}_1 - \vec{\xi}_2)\quad \wedge \quad \frac{d^2 \vec{\xi}_2}{d\tau^2}=\frac{1}{|\vec{\xi}_2 - \vec{\xi}_1|^3}(\vec{\xi}_1 - \vec{\xi}_2)
\end{equation}

Ahora buscamos las condiciones iniciales de este nuevo sistema \eqref{g0_c4:eq.nuv}. No se enunciarán las anteriores condiciones para ahorrar espacio.
A partir de la misma regla de la cadena vemos:

\begin{align}
\vec{r}_1(0)=&\vec{\xi}_1(0)\cdot R_0\\
\vec{\xi}_1(0)=&\vec{0}\\
\frac{d\vec{r}_1}{dt}(0)=&\frac{d\vec{\xi}_1}{d\tau}\cdot\sqrt{Gm_1/R_0^3}\\
\frac{d\vec{\xi}_1}{d\tau}(0)=&-\frac{\mu V_0} {\sqrt{Gm_1/R_0^3}}\hat{x}
\end{align}
Ahora para $m_2$:
\begin{align}
\vec{r}_2(0)=&\vec{\xi}_2(0)\cdot R_0\\
\vec{\xi}_2(0)=&\hat{y}\\
\frac{d\vec{r}_2}{dt}(0)=&\frac{d\vec{\xi}_2}{d\tau}\cdot\sqrt{Gm_1/R_0^3}\hat{x}\\
\frac{d\vec{\xi}_2}{d\tau}(0)=&\frac{V_0} {\sqrt{Gm_1/R_0^3}}\hat{x}
\end{align}
Entonces la velocidad normalizada la describiremos como: $\tilde{V}_0=\frac{V_0} {\sqrt{Gm_1/R_0^3}}$
\section{Método del salto de la rana}
Para utilizar este método se va a evaluar las funciones de manera intercalada, en pasos de $\Delta \tau$, para la velocidad en pasos semi-enteros, mientras que la posición en pasos enteros.
\begin{align}
\frac{d\vec{\xi}_i}{d \tau}\left(\tau+\frac{\Delta \tau}{2}\right)=&\frac{d\vec{\xi}_i}{d \tau}(\tau)+\frac{\Delta \tau}{2}\frac{d^2\vec{\xi}_i}{d \tau^2}(\tau)\\
\vec{\xi}_i(\tau+\Delta \tau)=&\vec{\xi}_1(\tau)+\Delta \tau\frac{d\vec{\xi}_i}{d \tau}\left(\tau+\frac{\Delta \tau}{2}\right)
\end{align}

\section{Aplicación}
Se escribe un script de Python para resolver numéricamente el sistema de ecuaciones \eqref{g0_c4:eq.nuv} usando el método del salto de rana. Considerando las cantidades normalizadas como $\mu=1/2$ y $\tilde{V}_0=1/4$. Para esto, se empieza definiendo la función de aceleración unificada para ambas ecuaciones.

\begin{minted}{python}
def aceleracion(d,norma,mu=-1):	#Se establece la acelarción en función de la posición, la norma y mu.
    a=-mu*(d)/(norma**3)
    return a
\end{minted}
Tras ello se ingresan las condiciones iniciales, se establecen los saltos de tiempo y los tamaños de los arreglos donde se registran las posiciones y velocidades de las masas. 
\begin{minted}{python}
#Parámetros para la simulación numérica.
dt = 0.01                       #Paso de tiempo.
pasos = 1000                     #Número de pasos de tiempo.
muu=0.5                         #Definimos la relacion entre las masas.
v0n=0.25                        #Determinamos la velocidad normalizada.
r1 = np.empty((pasos,2))        #Guardamos espacio para registrar las posiciones y velocidades de cada masa.
v1 = np.empty((pasos,2))
r2 = np.empty((pasos,2))
v2 = np.empty((pasos,2))
#Se definen las condiciones iniciales.
r1[0] = np.array([0.0,0.0])
v1[0] = np.array([-muu*v0n,0.0])
r2[0] = np.array([0.0,1.0])
v2[0] = np.array([v0n,0.0])
\end{minted}

Luego se realiza el ciclo donde se utiliza el salto de la rana para calcular las velocidades un semi-paso hacia adelante. Con éstas calcular las posiciones en un paso completo y luego avanzar en las velocidades la otra mitad del paso, para luego repetir el ciclo.

\begin{figure}
\centering
\includegraphics[scale=0.75]{../img/graf_control_04.png} 
\caption{Gráfico de las trayectorias de las masas.}
\label{g0_c4:fig:tray}
\end{figure}

Con esto se grafica las trayectorias de las masas en un intervalo de 10 unidades de tiempo normalizado \eqref{g0_c4:fig:tray}.

\section{Análisis de resultados}

Gracias a la generalidad de la ecuación de aceleración de gravedad, existen múltiples fuentes bibliográficas con las cuales revisar o comparar los resultados. Se ha de considerar como se realizaron los cálculos para dos dimensiones y como esto puede conducir a confusión a la hora de programar el script. Se compararon resultados con otros scripts de colegas y los resultados fueron similares.

\section*{Conclusiones}

Este control resultó ser muy desafiante para realizarlo solo en 45 minutos. Aun así mediante el uso de normalización de la ecuación, herramientas complementarias de Python y métodos numéricos de resolución de ecuaciones diferenciales, como lo es el método del salto de la rana, se lograron obtener resultados congruentes con la documentación.
\end{document}
