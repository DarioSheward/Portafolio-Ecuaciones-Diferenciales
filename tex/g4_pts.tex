\documentclass[../portafolio.tex]{subfiles}

\begin{document}

\chapter{Puntos de equilibrio en un sistema de ecuaciones diferenciales }
\label{g4_ej10}
\hfill \textbf{Fecha de la actividad:} 26 de noviembre de 2024

\medskip

A través de este capítulo, se determinan los puntos de equilibrio de un sistema de ecuaciones diferenciales ordinarias, donde ambas dependen del tiempo. Además, se revisa el comportamiento de estos puntos y se comparan con la dinámica representada en las gráficas de las soluciones numéricas del sistema, realizadas con el método de Runge-Kutta de cuarto orden.

\section*{Objetivos}
\begin{itemize}
\item Determinar los puntos de equilibrio de un sistema de ecuaciones diferenciales.
\item Determinar la estabilidad de los puntos de equilibrio.
\item Resolver numéricamente el sistema de ecuaciones.
\end{itemize}
Según el sistema de ecuaciones diferenciales planteado \eqref{g4_ej10:sisedo}.

\begin{equation}\label{g4_ej10:sisedo}
\frac{dx}{dt}=y-y^3 \quad \frac{dy}{dt}=-x-y^2
\end{equation}
\section{Puntos de equilibrio}
Para determinar los puntos de equilibrio del sistema de ecuaciones debemos buscar donde las variaciones de los términos sea cero, es decir, igualar las derivadas a cero.
\begin{align}
0=y(&1-y^2)\label{g4_ej10:pts0}\\ \intertext{Entonces los puntos de equilibrio tendrán $y=0$ , $y=1$ e $y=-1$, ahora reemplazamos en la otra ecuación y determinamos que los puntos de equilibrio serán los siguientes pares (x,y):}
(-1,-1),\quad (0,0)&, \quad y \quad (-1,1)
\end{align}

\section{Estabilidad de los puntos}

Se utiliza el método de las perturbaciones lineales \citep{ODE,Bender1978} para determinar la estabilidad de los puntos de equilibrio. Para esto los puntos de equilibrio serán $x=x^*$ e >$y=y*$ reemplazaremos por $x=x^* +\delta x$ y $y=y^*+ \delta y$, considerando que $\delta x$ y  $\delta y$ son pequeños.
Definamos el sistema como:
\[
\begin{cases}
\frac{d}{dt}(x^* + \delta x)=& f(x^* +\delta x, y^* + \delta y)\\
\frac{d}{dt}(y^* + \delta y)=& g(x^* +\delta x, y^* + \delta y)
\end{cases}
\]
Se expande en series de Taylor cada función del sistema, con centro en las perturbaciones y con el entorno definido como los puntos de equilibrio:

\begin{align}
f(x^* +\delta x, y^* + \delta y)=& f(x^*,y^*) + \frac{Df(x^*,y^*)}{Dt}\cdot (\delta x, \delta y)+...\\
\frac{Df(x^*,y^*)}{Dt}=& \left(
\begin{matrix}
\frac{df}{dx} \\
\frac{df}{dy} 
\end{matrix}\right) \cdot (\delta x, \delta y)
\end{align}
Al desarrollar esto como un sistema matricial con ambas ecuaciones. Sabemos que $f(x^*,y^*) = g(x^*,y^*)= 0$ al ser la imagen de un punto de equilibrio
\[ 
\frac{d}{dt} 
\begin{pmatrix} 
\delta x \\ 
\delta y 
\end{pmatrix} 
= 
\begin{pmatrix} 
\frac{df}{dx} & \frac{df}{dy}  \\ 
\frac{dg}{dx}  & \frac{dg}{dy}  
\end{pmatrix} 
\begin{pmatrix} 
\delta x \\ 
\delta y 
\end{pmatrix} 
\]
\[ 
\frac{d}{dt} 
\begin{pmatrix} 
\delta x \\ 
\delta y 
\end{pmatrix} 
= 
\begin{pmatrix} 
0 & 1-3y^2  \\ 
-1  & -2y  
\end{pmatrix} 
\begin{pmatrix} 
\delta x \\ 
\delta y 
\end{pmatrix} .
\]
Ahora, con el sistema linealizado, evaluamos los puntos de equilibrio en la matriz jacobiana y luego determinamos sus valores propios.

Sea la matriz jacobiana 
\[ 
J
= 
\begin{pmatrix} 
0 & 1-3y^2  \\ 
-1  & -2y  
\end{pmatrix} 
.
\]

Para el punto de equilibrio $(-1, -1)$:
\[ 
J_{(-1,-1)}
= 
\begin{pmatrix} 
0 & -2  \\ 
-1  & 2  
\end{pmatrix} 
\]
De valores propios
\begin{align}
\det(J_{(-1,-1)}-\lambda I) =& \lambda^2 -2\lambda - 2\\
\lambda= 1 \pm \sqrt{3}
\end{align}
Para el punto de equilibrio $(0,0)$:
\begin{align}
det(J_{(0,0)}-\lambda I) =& 
\begin{vmatrix}
-\lambda & 1 \\
-1 & -\lambda
\end{vmatrix}\\
det(J_{(0,0)}-\lambda I) =& \lambda^2 +1\\
\lambda=\pm i
\end{align}
Por último, para el punto $(-1,1)$:
\begin{align}
det(J_{(-1,1)}-\lambda I) =& 
\begin{vmatrix}
-\lambda & -2 \\
-1 & -2-\lambda
\end{vmatrix}\\
det(J_{(-1,1)}-\lambda I) =& \lambda^2 +2\lambda-2\\
\lambda= -1 \pm \sqrt{3}
\end{align}
Recordamos que las soluciones del sistema quedarán en forma de:
\begin{align}\label{g4_ej10:xx0}
x(t)=C_0 e^{\lambda_0 t} + C_1 e^{\lambda_1 t}\\ 
y(t)=D_0 e^{\lambda_0 t} + D_1 e^{\lambda_1 t} \label{g4_ej10:yy0}
\end{align}
%A partir de los valores de $\lambda$ al reemplazar por los puntos de equilibrio.
El método de perturbaciones lineales \citep{strogatz,guia_edos} establece que:
\begin{enumerate}
\item Si todos los valores propios del punto tienen partes reales positivas, el punto de equilibrio estable. 
\item Si ambas soluciones crecen exponencialmente, respecto a $t$ al ser la variable por la cual se deriva, el punto de equilibrio es inestable.
\item Si ambas soluciones decrecen exponencialmente respecto a $t$, el punto de equilibrio es estable tipo atractor.
\item Si ambas soluciones oscilan respecto a $t$, el punto de equilibrio es neutralmente estable (las trayectorias se mantienen cerca del punto de equilibrio).
\item Si una solución decrece exponencialmente y la otra oscila, el punto de equilibrio es un atractor y las trayectorias son en forma de espiral.
\item Si una solución crece y la otra decrece exponencialmente, el punto de equilibrio es una bifurcación o punto de silla.
\end{enumerate}

Entonces, las soluciones para el punto $(-1,1)$ son  
\begin{align}\label{g4_ej10:xx1}
x_{(-1,-1)}(t)&=C_0 e^{(1-\sqrt{3}) t} + C_1 e^{(1+\sqrt{3}) t}\\
y_{(-1,-1)}(t)&=D_0 e^{(1-\sqrt{3}) t} + D_1 e^{(1+\sqrt{3}) t}.\label{g4_ej10:yy1}
\end{align}
Las constantes $C_i$ y $D_i$ no están relacionadas con las constantes de \eqref{g4_ej10:xx0} y \eqref{g4_ej10:yy0}. En los siguientes casos, las constantes tampoco estarán relacionadas entre sí o con otras soluciones.

De las soluciones \eqref{g4_ej10:xx1} y \eqref{g4_ej10:yy1}, concluimos que el punto $(-1,-1)$ es una bifurcación.

En el caso del punto $(0,0)$ las soluciones son:
\begin{align}\label{g4_ej10:xx2}
x_{(0,0)}(t)=C_0 e^{-i t} + C_1 e^{i t}\\
y_{(0,0)}(t)=D_0 e^{-i t} + D_1 e^{i t}.\label{g4_ej10:yy2}
\end{align}
Por ende, el punto es neutralmente estable.

Para el caso del punto $(-1,1)$ las soluciones son:
\begin{align}\label{g4_ej10:xx3}
x_{(-1,1)}(t)=C_0 e^{(-1-\sqrt{3}) t} + C_1 e^{(-1+\sqrt{3}) t}\\
y_{(-1,1)}(t)=D_0 e^{(-1-\sqrt{3}) t} + D_1 e^{(-1+\sqrt{3}) t}.\label{g4_ej10:yy3}
\end{align}
Entonces, el punto $(-1,1)$ es una bifurcación.
\section{Solución numérica}
A continuación se muestra el script de Python utilizado para obtener las soluciones para múltiples puntos como condiciones iniciales, en la región de $x(0)\leq |2| \wedge y(0) \leq |2|$, para $0\leq t \leq 20$, y luego graficarlas en el espacio de fases $y(t)$ y  $x(t)$ (Figura \ref{g4_ej10:graf_0}). 
A continuación se explican los procesos del script.

En primera instancia se define el método de estimación numérica de ecuaciones diferenciales. En este caso se aplicó Runge-Kutta de orden 4, adaptado a dos ecuaciones con dos condiciones iniciales. Luego se definió el sistema de ecuaciones y los puntos usados como condiciones iniciales. Para estos últimos se aplicó un ajuste hacia el centroide del arreglo. Sea
\begin{equation}
x_{i}^{nuevo}=x_i + \alpha(C_x - x_i),
\end{equation}
donde $x$ es el arreglo de datos a ajustar, $C_x$ el centroide del conjunto, sea este el promedio de las coordenadas del arreglo, $\alpha$ es una proporción de cuanto desea desplazar el punto respecto a su distancia con el centroide, mientras i es el índice del dato en el arreglo. Finalmente, se aplica el método para cada condición inicial y se grafica.
\begin{minted}{python}
def runge_kutta_4(sistema, x0, y0, t0, tf, dt):	# se define el método de Runge Kutta de cuarto orden para un sistema de ecuaciones, ya que hay dos condiciones iniciales.
    n-pasos = int((tf - t0) / dt) + 1	#Se determina la cantidad de pasos dt que caben en el intervalo t.
    t =np.linspace(t0, tf, n-pasos)  	#se crea un arreglo de tiempo de datos equiespaciados, de tamaño determinado por n-pasos.
    x = np.zeros(n-pasos) 
    y =np.zeros(n-pasos) 

    x[0]= x0
    y[0]= y0
    
    for i in range(1, n-pasos):
        k1x, k1y =sistema(x[i- 1], y[i- 1])	#El método de Runge-Kutta de cuarto orden realiza 4 estimaciones de cual será el siguiente valor de la variable. Cada estimación la hace a partir de la anterior, excepto la primera, la ual es eqiuvalente al método de Euler.
        k2x, k2y=sistema(x[i -1] + dt* k1x / 2, y[i - 1] + dt * k1y/ 2)
        k3x, k3y=sistema(x[i -1] + dt * k2x/ 2, y[i- 1]+ dt* k2y/ 2)
        k4x, k4y=sistema(x[i -1] + dt * k3x, y[i - 1] + dt * k3y)

        x[i]= x[i -1] + (dt/6) * (k1x+2 * k2x+2 * k3x + k4x)	#Las estimaciones se ponderan y se determina la estimación final.
        y[i]= y[i -1] + (dt /6) *(k1y+ 2* k2y +2* k3y +k4y)
    return t, x, y 	# El método retorna los tiempos en los que se evaluó, el arreglo de la trayectoria en la componente x, y lo mismo para la componente y.
\end{minted}
\begin{minted}{python}
def sistema(x,y):	#Se define el sistema de ecuaciones.
    dx=y-y**3
    dy=-x-y**2
    return dx, dy
condiciones_iniciales = []	#Se preparan las distintas condiciones iniciales. Se realiza un proceso para desplazar los puntos hacia su centroide, es decir, acercarlos al origen, para observar mejor su comportamiento.
xs=np.linspace(-2,2,4)
ys=np.linspace(-2,2,4)
pnts_extremos=[xs[0],xs[-1],ys[0],ys[-1]]	
centroide=np.mean(xs[1:-1], axis=0)
proporcion=0.3
xs=xs +proporcion*(centroide-xs)
ys=ys +proporcion*(centroide-ys)
xs=np.concatenate([[pnts_extremos[0]], xs, [pnts_extremos[1]]])
ys=np.concatenate([[pnts_extremos[2]], ys, [pnts_extremos[3]]])
for i in xs:
    for n in ys:
        condiciones_iniciales.append((i,n))	#Se entrega un arreglo de pares ordenados que servirán como condiciones iniciales.
\end{minted}

\begin{minted}{python}
for x0, y0 in condiciones_iniciales:	#Se aplica el método para cada condición inicial del sistema y se grafica.
    t, x, y = runge_kutta_4(sistema, x0, y0, 0, 20, 0.01)
    plt.scatter(x[0],y[0], marker="2")
    plt.plot(x, y, label=f"$x_0=${round(x0,3)}, $y0_=${round(y0,3)}")
\end{minted}
\begin{figure}
\centering
\includegraphics[scale=0.75]{../img/g4_pnt_0.png}
\caption{Gráfico de fases de $y(t$) y $x(t)$. \label{g4_ej10:graf_0}}
\end{figure}
Las soluciones graficadas, en la figura \ref{g4_ej10:graf_0}, comienzan en $t=0$ (indicado con una marca de 3 líneas), cada una con un color característico. La disposición de los puntos donde empieza es simétrica, más no equiespaciada, debido al ajuste acercándolos al centroide.

En cuanto al comportamiento cercano a los puntos de equilibrio, revisamos la figura \ref{g4_ej10:graf_0}. El punto $(0,0)$ causa un comportamiento muy claro en la dinámica del sistema. Ejemplo de esto son las trayectorias de los puntos $(-0.4467,1.4)$ y $(0.44467,2)$, las cuales sufrieron desviaciones de apariencia cuasi-elíptica, hasta acercarse al punto de equilibrio $(-1,1)$, al ser este último un punto silla claramente se puede observar como por el lado que se acercan las trayectoriaa bajo estudio, éstas salen despedidas, alejándose de este. Por lo que podemos estimar que es en ese lado donde el punto causa comportamiento inestable. Mientras el punto $(-2,-1.4)$, en un comienzo, tiende a acercarse a este punto rosado, tras un tiempo acercándose, también presenta una desviación hacia el exterior del cuadro.

Volviendo al punto $(0,0)$, las trayectorias iniciadas en $(-0.467,-0.467)$ y $(-0.467,0.467)$ presentan un comportamiento, nuevamente, cuasi-elíptico. En estos casos se mantiene tal comportamiento, seguramente debido a la repulsión a los puntos silla cercanos. Comparemos este comportamiento con un punto simétrico con el eje y, $(0.467,0.467)$, él cuál también tiende a rodear el origen del gráfico, sin embargo, al acercarse a $(-1,-1)$, es disparado lejos de la órbita generada para otros puntos cercanos al $(0,0)$. 

El caso del punto $(-1,1)$ tiene un comportamiento similar a la otra bifurcación, es más, las trayectorias de los puntos de $x<0 \wedge y>1$ nos permiten observar mejor donde el punto genera atracción y donde repulsión. 

\section{Análisis de resultados}
Se logró determinar puntos de equilibrio, reconocer su comportamiento, de forma analítica. Éstos concordaron con las soluciones numéricas, presentando comportamientos dentro de lo expresado en las ecuaciones. Si bien las soluciones en el gráfico son limitadas, por razones de eficiencia en los tiempos de cálculo y saturación de información en la imagen. Aun así se adaptó la selección de condiciones iniciales en favor de facilitar la interpretación y toma de conclusiones a través de esta representación. Además, se adjunta un apéndice con una gráfica con más soluciones (\ref{g4_ej10:graf_1}).
\section*{Conclusiones}

En este ejercicio se determinan los puntos de equilibrio y su estabilidad mediante el método de las pertubaciones lineales. Éstos son comprobados graficamente a través de las soluciones numéricas del sistema de ecuaciones diferenciales ordinarias, utilizando el método de Runge-Kutta de cuarto orden. Además se realizo un ligero ajuste a las condiciones iniciales para obtener una representación más informativa del sistema. Así se logró estudiar el sistema de manera efectiva.
\section*{Agradecimientos}
Definitivamente, sin Steve Brunton, y sus videos, esto no hubiera sido posible. Además, muchas gracias al Dr. Roberto Navarro, por recibirme y resolver dudas que llevaban ahí por demasiadas horas. Me quedó muy claro que debería hacer un portafolio de ecuaciones diferenciales.

\begin{figure}
\centering
\includegraphics[scale=0.5]{../img/g4_pnt_1.png}
\caption{Apéndice de gráfico de fases de $y(t$) y $x(t)$, con más soluciones.} \label{g4_ej10:graf_1}
\end{figure}
\end{document}
