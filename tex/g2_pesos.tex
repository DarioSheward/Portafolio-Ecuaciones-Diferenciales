\documentclass[../portafolio.tex]{subfiles}

\begin{document}

\chapter{Determinación de pesos de un esquema de derivación}
\label{g2_ej6}
\hfill \textbf{Fecha de la actividad:} 25 de noviembre de 2024

\medskip

Durante este capítulo se revisa la generalización de un esquema de 3 nodos con peso y se revisan los pesos para la exactitud de la regla para ciertas funciones. Luego, también se comprueba el caso específico donde los nodos se encuentran a distancias iguales y el caso del esquema generalizado para los números complejos.
\section*{Objetivos}
\begin{itemize}
\item Determinar los pesos de un esquema de derivación de 3 nodos exacta para polinomios de hasta grado 2.
\item Demostrar como la el esquema de derivación de 3 nodos es equivalente al esquema de derivada centrada equiespaciada.
\item Determinar los pesos de un esquema de derivación de 3 nodos exacta para funciones complejas.
\end{itemize}

Todos los esquemas de derivación numérica centrada se pueden escribir en forma general como: 
\begin{equation} \label{g2_ej6:condicion}
f'(x_n)=\sum_{i=n-N}^{n+N} w_i f(x_i) , 
\end{equation}
donde $w_i$ es cada nodo que se le asigna a la imagen de $x_i$. En este caso, el esquema considera $N$ nodos a la izquierda y a la derecha de
$x_n$ , por lo que requiere evaluar $f (x_i)$ por cada uno de los $2N + 1$ nodos.

\section{Determinación de $w_i$}

Consideremos 3 se conocen tres puntos de una función, no necesariamente equiespaciados, $x_a$, $x_b$ y $x_c$. Queremos conocer la derivada de $x_b$.

Expandamos en serie de Taylor \citep{Stewart2001} las imagenes de los puntos $x_a$ y $x_c$ centradas en $x_b$ y de entorno $\Delta_a=x_a -x_b$ y $\Delta_c=x_c - x_a$ respectivamente.

\begin{align}
f(x_b)=&f(x_a)-f'(x_b)\Delta_a+\frac{f''(x_b )}{2!}\Delta_a^2-\frac{f''(\xi)}{3!}\Delta_a^3\\
f(x_b)=&f(x_c)+f'(x_b)\Delta_c+\frac{f''(x_b )}{2!}\Delta_c^2+\frac{f''(\xi)}{3!}\Delta_c^3
\end{align}

Así reemplazando en \eqref{g2_ej6:condicion} 
\begin{align}
f'(x_b)=&  w_a\left(f(x_b)+f'(x_b)\Delta_a-\frac{f''(x_b )}{2!}\Delta_a^2...\right)+w_b f(x_b) + w_c\left(f(x_b)-f'(x_b)\Delta_c-\frac{f''(x_b )}{2!}\Delta_c^2...\right)\label{g2_ej6:swiftie}
\end{align}

Ahora solucionamos $f(x)=1$, $f(x)=x$ y $f(x)=x^2$:
\[
\begin{cases}
0=	& w_a+w_b  + w_c\\
1=	&w_a\left(x_b+\Delta_a\right)+w_b x_b + w_c\left(x_b-\Delta_c\right) \\
2x_b=&w_a\left(x_b^2-2x_b\Delta_a-\Delta_a^2\right)+w_b x_b^2 + w_c\left(x_b^2+2x_b\Delta_c-\Delta_c^2\right),
\end{cases}
\]
y queda el sistema de ecuaciones. Ahora que sabemos que $w_b=-(w_a +w_c)$, podemos reemplazar en la segunda ecuación para ver que
\begin{align}
1= &w_a\Delta_a   - w_c\Delta_c \\ \intertext{y en la segunda ecuación}
2x_b=&-(\Delta_a^2 w_a+ \Delta_c^2 w_c)
\end{align}
Consideramos $w_a=-\frac{1-w_c \Delta_c}{\Delta_a}$ y $w_c=\frac{1+w_a \Delta_a}{\Delta_c}$
\begin{align}
w_a=&\frac{ 2x_b-\Delta_c }{\Delta_a^2 +\Delta_a \Delta_c}.\label{g2_ej6:peso1}\\ \intertext{Entonces seguimos buscando los otros pesos.}
w_c=& \frac{\Delta_a + 2x_b}{\Delta_c^2 +\Delta_a \Delta_c}\label{g2_ej6:peso2}\\
w_b=& -\frac{ 2x_b-\Delta_c }{\Delta_a^2 +\Delta_a \Delta_c}-  \frac{\Delta_a + 2x_b}{\Delta_c^2 +\Delta_a \Delta_c}\label{g2_ej6:peso3}
\end{align}
Sean \eqref{g2_ej6:peso1}, \eqref{g2_ej6:peso2} y \eqref{g2_ej6:peso3} los pesos de la expresión \eqref{g2_ej6:condicion} siendo un esquema exacto para polinomios de grado menor a 3.
\section{Derivación de tres nodos equiespaciados}

Con los resultados de la sección anterior, donde determinamos los pesos para el esquema de derivación de 3 nodos, ahora consideraremos que $h=-\Delta_a=\Delta_c$. Entonces para un esquema
\begin{equation}\label{g2_ej6:eq:tanto}
f'(x_b)=w_{a}f(x_{a}) +w_b f(x_b) + w_{c}f(x_{c}).
\end{equation}
Ahora simplemente reemplazamos en \eqref{g2_ej6:peso1}, \eqref{g2_ej6:peso2} y \eqref{g2_ej6:peso3}.
\begin{align}
w_a=&\frac{2x_b-h}{2h^2}\\
w_c=&\frac{h+2x_b}{2h^2}\\
w_b=&0
\end{align}
Volviendo a \eqref{g2_ej6:eq:tanto}:
\begin{equation}\label{g2_ej6:eq:centrada}
f'(x_b)=\frac{2x_b-h}{2h^2}f(x_{a})  + \frac{h+2x_b}{2h^2}f(x_{c}).
\end{equation}

Dado que sabemos que el intervalo es simétrico podemos obviar el término $x_b$. Entonces queda 
\begin{equation}
f'(x_b)=\frac{f(x_c)- f(x_a)}{2h}.
\end{equation}
Sea esto equivalente a 
\begin{equation}\label{g2_ej6:ahoara}
f'(x_b)=\frac{f(x_b +h)- f(x_b -h)}{2h},
\end{equation}
el esquema de derivación centrada de segundo orden.
\section{Determinación de los pesos del esquema para funciones complejas}
Repetiremos el proceso de encontrar los pesos pero con las funciones $f(x)=1$, $f(x)=e^{ix}$ y $f(x)=e^{-ix}$. Esta vez evitaremos la expansión en serie de Taylor y utilizaremos \eqref{g2_ej6:eq:tanto}.

\[
\begin{cases}
0=	& w_a+w_b  + w_c\\
ie^{ix_b}=	&w_a e^{ix_a}+w_b e^{ix_b} + w_c e^{ix_c} \\
-ie^{-ix_b}=&w_a e^{-ix_a} +w_b e^{-ix_b} + w_c e^{-ix_c}
\end{cases}
\]

Nuevamente $w_b=-(w_a+w_c)$. 
\[
\begin{cases}
w_b=&-(w_a+w_c)\\
i-w_b= &w_ae^{-i\Delta_a}   + w_ce^{-i\Delta_c}\\
+i-w_b= &w_a e^{ i\Delta_a} + w_c e^{ i\Delta_c}
\end{cases}
\]
Sumando la segunda y tercera fila.
\begin{equation}
-2w_b=w_a (e^{i\Delta_a}+e^{-i\Delta_a}) +w_c (e^{i\Delta_c}+e^{-i\Delta_c})
\end{equation}

Al tratarse del conjugado de un número complejo, quedan solo las partes reales.
\begin{align}\label{g2_ej6:complejos}
-w_b=w_a cos(\Delta_a) + w_c cos(\Delta_c)
\end{align}
Así todos los pesos son puramente reales.

\section{Análisis de resultados}
Es interesante ver como al tomar generalidades como estas es posible observar como los términos se alinean para formar lo más especifico. Si consideramos intervalos equidistantes en el caso de los números complejos \eqref{g2_ej6:complejos}, podemos notar como $w_b=0$ dejando la relación de los otros pesos como $w_a=w_c$, resultando en el mismo esquema de derivación centrada \eqref{g2_ej6:ahoara}.
\section*{Conclusiones}
Mediante álgebra y lógica es posible generalizar este tipo de métodos numéricos, sin embargo, no es un proceso simple. Deben proponerse las condiciones correctas a vistas de lo que se quiere mostrar. Lo último resulto un proceso difícil de afrontar en un comienzo, aun así, se logró, al igual que los objetivos propuestos.
%\section*{Agradecimientos}
\end{document}
