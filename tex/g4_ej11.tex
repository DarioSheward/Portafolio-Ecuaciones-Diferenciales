\documentclass[../portafolio.tex]{subfiles}

\begin{document}

\chapter{Ecuación logística}
\label{g4_ej11}
\hfill \textbf{Fecha de la actividad:} 29 de noviembre de 2024

\medskip

En este capítulo se revisa la función logística para solucionarla con métodos analíticos y numéricos, para luego compararlos y analizar los resultados. Así, se obtuvo información sobre los puntos de equilibrio y la estabilidad de éstos.

\section*{Objetivos}
\begin{itemize}
\item Resolver analíticamente la ecuación logística.
\item Analizar los resultados analíticos con apoyo gráfico obtenido mediante métodos numéricos.
\end{itemize}
La ecuación logística es el modelo más simple para describir el crecimiento poblacional de una especie:
\begin{equation}\label{g4_ej11:logistica}
\frac{dN}{dt}=r N \left(1 -\frac{N}{K}\right),
\end{equation}
donde $N=N(t)$ es la población total a tiempo $t$, $r>0$ es la tasa de crecimiento, y sea $K>0$  la capacidad de carga.
\section{Solución analítica}
Empezaremos con un sacrilegio matemático. Separamos las diferenciales.
\begin{align}
dN=&rN\left(1 -\frac{N}{K}\right)dt\\
rdt=&\frac{1}{N\left(1 -\frac{N}{K}\right)}dN\\
rdt=&\frac{K}{N\left(K -N\right)}dN\\
rdt=&\left( \frac{1}{N}+ \frac{1}{\left(K -N\right)}\right)dN \\ \intertext{Ahora integramos, solo representaremos una constante para evitar sobrecargar la visual de la ecuación.}
r t +C =& ln|N| - ln|K-N|	\\
rt+C= & ln\left|\frac{N}{K-N} \right|
\end{align}
Aplicando la función exponencial llegamos a un resultado interesante:
\begin{align}
\frac{N}{K-N} =& e^{rt+C}.\\ \label{g4_ej11:util_0} \intertext{Consideramos $e^C = \tilde{C}$ para poder determinarle utilizando la condición inicial $N(0)=N_0$.}
\frac{N_0}{K-N_0} =& \tilde{C} \\ \intertext{Volviendo a \eqref{g4_ej11:util_0}:}
\frac{N}{K-N} =& \frac{N_0}{K-N_0} e^{rt}.\\ \intertext{Sea la solución para $N(t)$:}
N(t)=&\frac{KN_0e^{rt}}{K-N_0(1-e^{rt})}.
\end{align}
Ahora si queremos ver soluciones hemos de determinar ciertas condiciones. Estableceremos una relación de $N_0/K=3$, mientras los valores de $r$ y $N_0$ variarán de 1 a 3 y a 5, mientras t se graficará en el intervalo $t \in [0,5]$. Esto resulta en el gráfico \ref{g4_ej11:11_graf}. El script para esta figura fue:
\begin{minted}{python}
def logi(t,N0,r,K): #Se define la función analítica.
    return (K*N0*np.exp(r*t))/(K-(N0*(1-np.exp(r*t))))
#Condiciones iniciales.
t=np.linspace(0,5,100)  #Intervalo de tiempo.
N0s=np.linspace(1,5,3)  #Poblaciónes iniciales.
rzon=3      #Razón N0/K.
rs=np.linspace(1,5,3)      #Valores que tomará r.

for N0 in N0s:  # Se define el ciclo donde avanzan los valores de N0.
    K=N0/rzon       #Se define K según la condición inicial y la razón definida.
    for r in rs:    #Ciclo donde avanzan los valores de r.
        trayectoria=logi(t,N0,r,K)  #Se retorna el recorrido de N(t).
        plt.plot(t,trayectoria, label=f'$N_0=${N0}, $r=${r}')   #Se grafican las trayectorias.
\end{minted}
\begin{figure}
\centering
\includegraphics[scale=0.75]{../img/g4_ej11_0.png} 
\caption{Representación de la cantidad de población según el tiempo, a partir de la condiciones impuestas.}\label{g4_ej11:11_graf}.
\end{figure}
\section{Normalización}
En esta sección se dejará la función \eqref{g4_ej11:logistica} normalizada para luego replicar las condiciones de la figura \ref{g4_ej11:11_graf} y comparar los resultados.

Primero consideramos las normalizaciones $n=N/K$ y $\tau=rt$ Entonces reemplazando en \eqref{g4_ej11:logistica}:
\begin{align}
\frac{d(nK)}{d(\tau /r)}=& r(nK)(1-n)\\
\frac{dn}{d\tau}=& n(1-n). \label{g4_ej11:norma}\\ \intertext{Determinando su solución como:}
\frac{1}{n(1-n)}dn=d\tau \\
\frac{1}{n}+\frac{1}{1-n}dn=&d\tau\\
ln\left|\frac{n}{1-n}\right| = \tau + C\\
\frac{n}{1-n}= \tilde{C} e^{\tau}\\ \intertext{Se a la condición inicial normalizada:}
\frac{n_0}{1-n_0}=&\tilde{C}.\\ \intertext{Entonces, la ecuación normalizada quedará como:}
\frac{n}{1-n}=&\frac{n_0}{1-n_0}\\
n(\tau)=& \frac{n_0e^{\tau}}{1+n_0(e^{\tau}-1)}.
\end{align}
Ahora para recrear las condiciones de la sección anterior consideramos $n=3$ y $\tau$ representará solo uno de los valores de r, seleccionamos $\tau=1$ para que coincida con solo el menor grupo de las curvas de \ref{g4_ej11:11_graf}, la cual presenta los mismos comportamientos que las trayectorias de $r=1$ en la figura \ref{g4_ej11:11_graf_1}.

Entonces al graficar resulta la figura \ref{g4_ej11:11_graf_1}. Cuya realización se logró a través de este script:
\begin{minted}{python}
def logi(t,n0):     #Se define la solución analítica de la ecuación normalizada.
    return (n0*np.exp(t))/(1-(n0*(1-np.exp(t))))
#condiciones iniciales
t=np.linspace(0,5,100)  #Se define el intervalo de tiempo normalizado a recorrer.
n0s=np.linspace(1,5,3)  #Se define que condiciones iniciales se graficarán.

for n0 in n0s:  #Ciclo donde avanzan las condiciones iniciales.
    trayectoria=logi(t,n0)     #Se registra la trayectoria de las curvas.
    plt.plot(t,trayectoria, label=f'$n_0=${n0}')    #Se grafica la solución para la condición inicial elegida.
\end{minted}
\begin{figure}
\centering
\includegraphics[scale=0.75]{../img/g4_ej11_1.png} 
\caption{Población normalizada contra tiempo normalizado. }\label{g4_ej11:11_graf_1}.
\end{figure}


\section{Puntos de estabilidad}
Con la ecuación normalizada \eqref{g4_ej11:norma}, se solicita solucionar esta expresión con el método de Euler y el de Runge-Kutta de orden 2, para condiciones iniciales de $n_0 \in [0,3]$ y tiempos $\tau <10$. Al graficar lo solicitado, obtenemos el gráfico \ref{g4_ej11:11_graf_2}.

El script realizado para esta figura es:
\begin{minted}{python}
#El tiempo normalizado se nombrará como t, para facilitar la programación.
def dn(n):  #Se define la derivada de n normalizada para ser utilizada en los siguientes métodos.
    return n-n**2
def Euler(t,n0,dn): #El método de Euler requiere conocer el primer punto, conocer la derivada de la variable a estimar y los pasos de tiempo.
    dt=t[1]     #Se define el tamaño de los pasos de tiempo. Se toma el índice 1 debido a que t[0]=0.
    n=np.empty(len(t))      #Se prepara el arreglo para guardar los valores de n mientras pasa t.
    n[0]=n0                 #Se define la condición inicial.
    for i in range(len(t)-1):   #Se realiza el ciclo por el tamaño del arreglo t-1. Para evitar pasarse del tamaño de los arreglos.      
        n[i+1]=n[i]+dt*dn(n[i])  #Se estima la siguiente posición a partir de la derivada y el paso de tiempo.
    return n    #Se regresa los valores de n a través del intervalo de tiempo.
def RungeKutta2(t,n0,dn):   #El método de RK-2 requiere conocer el primer punto, conocer la derivada de la variable a estimar y los pasos de tiempo.
    dt=t[1] 
    n=np.empty(len(t))
    n[0]=n0
    for i in range(len(t)-1):   #Este método define dos sumandos a partir de las variaciones que debería sufrir n y n+1(provisoriamente) en un período dt, para promediarlos y determinar definitivamente n+1.
        k1=dt*dn(n[i])
        k2=dt*dn(n[i]+k1)
        n[i+1]=n[i]+0.5*(k1+k2)
    return n    #Se regresa los valores de n a través del intervalo de tiempo.

#condiciones iniciales
t=np.linspace(0,10,10000)   #Se define el intervalo de tiempo normalizado.
n0s=np.linspace(0,3,10)     #Se deciden tomar 10 valor equiespaciados, entre 0 y 3, como condiciones iniciales.

for n0 in n0s:  #Ciclo donde avanzan las condiciones iniciales.
    n_euler = Euler(t,n0,dn)    #Se aplican los métodos para cada condición inicial.
    n_RK =RungeKutta2(t,n0,dn)
    plt.plot(t,n_euler, label=f'E-$n_0=${round(n0,3)}') #Se grafican todas las soluciones.
    plt.plot(t, n_RK, label=f'RK-$n_0=${round(n0,3)}')  #En este caso, ambas son precisas en este intervalo, por lo que no se logra diferenciar un método del otro.
\end{minted}

Observando la figura \ref{g4_ej11:11_graf_2} podemos observar que las soluciones graficadas tienden a los puntos $n=0$ y $n=1$.	
Entonces, igualamos la variación de población a $0$ en la ecuación \eqref{g4_ej11:norma}, para determinar los puntos de estabilidad.
\begin{align}
n(1-n)=0\\
n=0 \wedge n=1
\end{align}
Sabiendo que estos dos puntos son de equilirbio, ahora los clasificaremos según su estabilidad. Esto se suele ver estudiando las soluciones para cada condición inicial. Sin embargo es bastante claro como estos puntos son de estabilidad, ya que se conservan durante el tiempo.

\begin{figure}
\centering
\includegraphics[scale=0.75]{../img/g4_ej11_3.png} 
\caption{Población normalizada contra tiempo normalizado, para determinar puntos de equilibrio. }\label{g4_ej11:11_graf_2}.
\end{figure}

\section{Análisis de resultados}
Las instrucciones detalladas de este ejercicio resultan parte vital de poder sacar obtener las soluciones esperadas, sin embargo, limitan los posibles casos extremos. Aun así se considero el graficar un diagrama de fases para determinar la estabilidad de los puntos, sin embargo, al tratarse de puntos estables, aquella representación no resultó muy informativa. De todas maneras, tal comportamiento se pudo representar en la figura \ref{g4_ej11:11_graf_2}. 
\section*{Conclusiones}
Durante el desarrollo de este ejercicio se utilizaron métodos analíticos  y numéricos para resolver la ecuación logística, tanto en su forma común como normalizada, graficando sus soluciones y encontrando sus puntos de estabilidad. Los últimos se comprueban analíticamente, para luego, realizar observaciones sobre sus propiedades de estabilidad. Sean estos todos los objetivos planteados, por lo que se considera un resultado exitoso.

\end{document}