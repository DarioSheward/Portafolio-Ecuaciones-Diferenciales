\documentclass[../portafolio.tex]{subfiles}

\begin{document}

\chapter{Error al estimar una función con series de Maclaurin }
\label{g1_ej20}
\hfill \textbf{Fecha de la actividad:} 6 de noviembre de 2024

\medskip
En este capítulo se desarrolla un script para comparar una función analítica con su aproximación numérica, a través de series de Maclaurin. Se estiman las capacidades numéricas del método revisando los errores relativos y absolutos registrados.
\section*{Objetivos}
\begin{itemize}
\item Estudiar el error absoluto y relativo entre la estimación a través de series de Maclaurin y la función evaluada.
\end{itemize}

En esta ocasión se estudiará el caso más simple de una serie de Maclaurin, $m(x)=e^x=\sum_{n=0}^{k} \frac{x^n}{n!}$, para realizar una estimación numérica de la serie se realizará un script de Python que calculo los términos de la serie hasta que algún término sea menor a una tolerancia definida. Luego, se compararán gráficamente y se determinarán sus errores absolutos y relativos para distintas tolerancias.

\section{Programación de la estimación}
El siguiente script de Python define una función exponencial descrita por la series de Maclaurin \eqref{eq:serieex}. Esta es una expansión de serie de Taylor centrada en $x=0$ (como la derivada de la función exponencial es igual a la derivada del exponente por de sí misma) entonces cada derivada evaluada en $x=0$ es $1$.\\
Volviendo al script, este funciona con una tolerancia variable, la cual reacciona ante la diferencia entre la función evaluada, $f(x)=e^x$, y la serie $m(x)=e^x=\sum_{n=0}^{k} \frac{x^n}{n!}$, donde $m$ sea la iteración donde $|f(x)-m(x)|<\varepsilon$, sea $\varepsilon$ la tolerancia elegida. En el código, la variable \texttt{j} acumula la sumatoria de la función $m$, la cual itera en el ciclo \texttt{while}, que depende de la tolerancia. Cuando la diferencia entre \texttt{j} y el $f(x)$ sea menor a la tolerancia, el valor del primero se guardará en el arreglo de \texttt{m}, cuando esto ocurra para todo el arreglo a evaluar, \texttt{x}, se retornará el arreglo \texttt{m}.
\begin{minted}{python}
def e(x,tol=0.01):
    m=np.empty(len(x))
    for i in range(len(x)):
        n=0
        j=0
        while np.abs(j-np.exp(x[i]))>tol:
            j=j+((x[i]**n)/fac(n))
            n=n+1
        m[i]=j
    return m
\end{minted}
\begin{equation}\label{eq:serieex}
e^x=\sum_{n=0}^{\infty}\frac{x^n}{n!}
\end{equation}
Luego se registran y grafican $f(x)$, $m(x)$, el error absoluto y el error relativo con dos tolerancias distintas.

\begin{figure}
\centering
\includegraphics[scale=0.75]{../img/expansionTaylor0.1.png} 
\caption{Gráfico comparando $e^x$ con su aproximación numérica de toleracia 0.1.}\label{fig:tol0.1}
\end{figure}

\begin{figure}
\centering
\includegraphics[scale=0.75]{../img/expansionTaylor0.01.png} 
\caption{Gráfico comparando $e^x$ con su aproximación numérica de tolerancia 0.01}\label{fig:tol0.01}
\end{figure}

\section{Análisis de Resultados}
El comportamiento de la función exponencial para los valores de $x<0$ es asintótico a la recta $y=0$. Se presenta un punto donde el valor de la función por sí mismo ya es menor a la tolerancia. A la vez las estimaciones también se vuelven diminutas, se van generando errores absolutos diminutos, sean estos errores los que considera la tolerancia, sin embargo, en comparación a la magnitud de los valores evaluados, representan hasta un 100\%. \\
Al disminuir la tolerancia se puede retrasar este fenómeno, incitando a un mayor número de iteraciones del ciclo del script. Aun así, al disminuir demasiado la tolerancia, se arriesga a la presencia de un underflow en alguno de los procesos de cálculo de error. La tolerancia de menor orden magnitud que se pudo graficar es la escogida para la figura \ref{fig:tol0.06}.
\begin{figure}
\centering
\includegraphics[scale=0.75]{../img/expansionTaylor1e-06.png} 
\caption{Gráfico comparando $e^x$ con su aproximación numérica de tolerancia 1e-6}\label{fig:tol0.06}
\end{figure}


\section*{Conclusiones}
Este ejercicio se pudo presentar como ocasión para revisar las capacidades numéricas de las series de Macluarin. Se logró presentar gráficos informativos y desarrollar un script que cumplió su propósito. Se obtuvieron resultados favorables, además de esperados.
\section*{Agradecimientos}
Agradecer a Rigoberto McPato que me escucho preguntarme (¿Por qué el gráfico no pasa por uno en $x=0$?).
\end{document}