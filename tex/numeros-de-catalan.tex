\documentclass[../portafolio.tex]{subfiles}

% Solo agregue paquetes en el preámbulo de ../portafolio.tex

\begin{document}

\chapter{Números de Catalán}
\label{g1_ej2}
\hfill \textbf{Fecha de la actividad:} 20 de octubre de 2024

\medskip

En este capítulo trabajaremos con los números de Catalán, mostrando y analizando algunas de sus propiedades. Revisando su definición analítica y comparándola con dos series, a través de representaciones gráficas. Se complementan estos resultados desarrollando las expresiones analíticamente, mostrando el comportamiento de las series, cuya visualización es limitada por la precisión simple.

\section*{Objetivos}
\begin{itemize}
\item Mostrar relación entre números de Catalán y series que la representan.
\item Analizar el comportamiento de series, a partir de los números de Catalán, a través de métodos numéricos.
\item Evaluar las capacidades de la precisión simple (32bits).
\end{itemize}
\section{Introducción}

Los números de Catalán son una secuencia de números naturales, descrita por primera vez por Leonard Euler (1707-1783), aunque Eugène Charles Catalan (1814-1894) es el epónimo de la suceción. Estos números han encontrado aplicaciones en la teoría de grafos, la teoría de juegos y la enumeración combinatoria, convirtiéndolos en una gran herramienta en varias áreas de la matemática. Por ejemplo: triangulaciones, conteo de caminos, secuencias balanceadas, entre otras.
\section{Series a estudiar}

Los números de Catalán se definen, para $n \geq 0$, como:
\begin{equation}\label{g1_ej1:eq:catalan}
C_n=\frac{(2n)!}{(n+1)!n!}
\end{equation}
Se nos pide demostrar que se pueden escribir como la serie:
\begin{align}\label{g1_ej1:eq:seriecat}
C_0=1&	&C_{n+1}=\frac{4n+2}{2n+2}C_n
\end{align}
Para $C_0$ podemos reemplazar en \eqref{g1_ej1:eq:catalan}:
\begin{align*}
C_0=\frac{0!}{1!0!}
\end{align*}
Podríamos cancelar los factoriales de $0$, sin embargo, primero revisemos si es un número con el que podamos trabajar
Según la definición del factorial \citep{GavilanGonzales2017}. Para cualquier $x\geq 0$
\begin{equation}
x!=\Gamma (x+1),
\end{equation}
donde:
\begin{equation}
\Gamma (u) = \int_{0}^{\infty}t^{u-1}e^{-t} \, dt,\footnote{Esta función se revisa a profundidad en el capítulo \ref{g4_ej10}.}
\end{equation}
si $x=0$, entonces $0!=\Gamma (1)$:
\begin{align*}
\Gamma (1) &= \int_{0}^{\infty}e^{-t} \, dt \\
\Gamma (1) &= \lim_{R\rightarrow \infty } -e^{-R} + e^{0}\\
\Gamma (1) &=1
\end{align*}
Sabiendo que podemos operar con $0!$, queda demostrado que $C_0=1$.\\

Ahora, para demostrar 
\begin{align*}
C_{n+1}=\frac{4n+2}{2n+2}C_n ,
\end{align*}
se empieza reemplazando en \eqref{g1_ej1:eq:catalan}.
\begin{equation}
C_{n+1}=\frac{(2(n+1))!}{(n+2)!(n+1)!}
\end{equation}

Reescribiremos el término a la derecha:
\begin{align*}
C_{n+1}=& \frac{(2n+2)!}{(n+2)!(n+1)!}\\
C_{n+1}=& \frac{(2n+2)(2n+1)(2n)!}{(n+2)(n+1)(n+1)!n!}\\
C_{n+1}=& \frac{2(n+1)(2n+1)}{(n+2)(n+1)}C_n\\
C_{n+1}=& \frac{4n+2}{2n+2}C_n
\end{align*}
Así queda demostrado.

\section{Comparación numérica de expresiones}
El enunciado del ejercicio nos solicita graficar con Python todos los números de Catalán menores a una cota, ingresada por un usuario, \texttt{M}. También se solicita que se utilice precisión simple (32bits) para calcular $C_n$. Tras esto debemos comparar los resultados de usar la definición \eqref{g1_ej1:eq:catalan} y la serie \eqref{g1_ej1:eq:seriecat}. Utilizaremos como cota $M>10^{15}$.\\

El código utilizado para el gráfico se ilustrará a continuación. Se inició definiendo el factorial, la definición \eqref{g1_ej1:eq:catalan} y la serie \eqref{g1_ej1:eq:seriecat} como funciones de Python, para luego poder operar con ellas.
\begin{minted}{python}
def d32(x): #Definimos precisión simple 
    return np.float32(x)
def numCatalan(n): #Definimos el primer método para los números de Catalán
    return d32(fac(2*n)/(fac(n+1)*fac(n)))
def numCatalan2(n): #Definimos el segundo método para calcular los números de Catalán
    return d32(((4*n)+2)*numCatalan(n)/(n+2))
\end{minted}

Se le permite ingresar al usuario un número como cota, M. Tras ellos se crean los arreglos donde se registrara cada número entero positivo, $n$, y su $C_n$. Además se determina que el primer valor de n=0. Para la serie \ref{g1_ej1:eq:seriecat} se define a mano $C_0 = 1$ tal como se hizo cuando se describió la serie.
\begin{minted}[escapeinside=||]{python}
M = d32(float(input("Ingresa el valor de M: ")))
n_ec1 = [] #A continuación se calcularán ambos métodos a partir de la misma n, por esto, para n=0, en la ec.1 se agrega C0, mientras que para la ec.2, calulará C1.
n_ec2 = [0]
Cn_ec1 = []
Cn_ec2 =[1]    
n = 0 #Cuando empezamos desde n=1, la ec.2 esta calculando C1 a partir de el valor de C0, por esto este arreglo.
\end{minted}
Se realiza un ciclo donde se calcula a través de cada expresión, registrándolos en arreglos distintos, mientras $n$ avanza de unidad en unidad. El ciclo termina cuando se llegue al $n$ donde $C_n \geq M$. 
\begin{minted}[escapeinside=||]{python}
while True: #Este es el proceso de cálculo donde se calcula desde n hasta un número cuyo "número de Catalán" 
    Cn1 = numCatalan(n)
    Cn2 = numCatalan2(n)
    if Cn1 >= M: 
        break
    n_ec1.append(n)
    Cn_ec1.append(Cn1)
    Cn_ec2.append(Cn2)
    n += 1
\end{minted}
Entonces ahora se prepara el gráfico de la figura \ref{g1_ej1:fig:catalan}. Se selecciona una escala logarítmica para el eje y para poder analizar el crecimiento mejor.
\begin{minted}{python}
plt.scatter(n_ec1, Cn_ec1, label='Ecuación 1',marker='.')
plt.scatter(n_ec2[:-1], Cn_ec2[:-1], label='Ecuación 2',marker='.') #Quitamos el último numeros ya que será mayor a M, lo cuál va contra la instrucción.
plt.xlabel('n')
plt.ylabel('Cn')
plt.title(f'Números de Catalán menores que {Cn1}')
plt.legend()
plt.yscale('log')  # Usar escala logarítmica para apreciar mejor lo que ocurra
plt.show()
\end{minted}
Para graficar la serie no consideramos el último cálculo ya que este será $C_{m+1}>M$. Sea $m$ el valor de $n$ donde se detiene el programa, para que $C_n$ no supere $M$.

\begin{figure}[ht]
\centering
\includegraphics[scale=0.7]{../img/catalan_graf.png}
\caption{Números de Catalán menores $10^{15}$.}
\label{g1_ej1:fig:catalan}
\end{figure}

\begin{figure}[ht]
\centering
\includegraphics[scale=0.7]{../img/catalan_2.png} 
\caption{Gráfico de serie a estudiada.}
\label{g1_ej1:fig:catalan2}
\end{figure}

\subsection{Estudio del comportamiento asintótico de una serie}
Por último se estudia la convergencia de la serie \eqref{g1_ej1:aestudiarcat}
\begin{equation}\label{g1_ej1:aestudiarcat}
C_n \approx \frac{4^n}{n^{3/2}\sqrt{\pi}}
\end{equation}
a partir de la definición de los números de Catalán. Como se puede ver en el gráfico \ref{g1_ej1:fig:catalan2}, la definición \eqref{g1_ej1:eq:catalan} es siempre menor a \eqref{g1_ej1:aestudiarcat}. A partir de esto, por criterio de comparación de series \citep{Stewart2001},

\begin{equation*}
\text{si} \quad \lim_{n\rightarrow \infty } C_n \quad \text{diverge, entonces la serie } \frac{4^n}{n^{3/2}\sqrt{\pi}}\quad \text{diverge.}
\end{equation*}

Ahora calculamos
\begin{equation*}
\lim_{n\rightarrow \infty } \frac{(2n)!}{(n+1)!n!} \quad \text{y concluimos que es divergente.}
\end{equation*}
Entonces implica que la serie \eqref{g1_ej1:aestudiarcat} es divergente. Lo cual coincide con el gráfico \ref{g1_ej1:fig:catalan2}, mostrando un comportamiento asintótico.

Para este gráfico se copió el archivo para la figura \ref{g1_ej1:fig:catalan} y se agregaron las siguientes líneas para definir arreglos de la serie.

\begin{minted}{python}
xx=np.arange(1,len(n_ec1))
yy = [] 
for i in range(len(n_ec1)-1):
    yy.append((4**int(xx[i])) / ((int(xx[i])**1.5)))
plt.scatter(xx,yy, label='Serie a estudiar', marker='.',c='purple')
\end{minted}

\section{Análisis de resultados}
En primer lugar, la equivalencia numérica entre la definición \eqref{g1_ej1:eq:catalan} y la serie \eqref{g1_ej1:eq:seriecat}, resulta en un gráfico con puntos, de cada expresión, indistinguibles uno del otro. Considerando que se tomó como limitación el usar precisión simple, podemos notar que esta no mermó la calidad de los resultados, como es de esperar considerando que la documentación \citep{IEEE754-2008} de \texttt{np.float32} soporta números hasta $10^{38}$ aproximadamente.
En el caso del segundo gráfico, notar que se calcularon tantos $n$'s como permite la precisión simple, alrededor de $10^{38}$. Esto, claramente, afecto a los puntos naranjos \eqref{g1_ej1:eq:seriecat} y no a los puntos azules \eqref{g1_ej1:eq:catalan}. Puede explicarse por el caracter binario de las operaciones realizadas, considerando que primero se realizan las multiplicaciones, donde seguramente ya habría superado la capacidad de la precisión simple, y luego se divide por el denominador de la expresión. 

\section*{Conclusiones}
A través de este ejercicio se indagó en caracteristicas analíticas y numéricas de los números de Catalán, reconociendo las capacidades de la precisión simple, mediante la comparación de métodos analíticas y numéricos. Destacar como el trabajo requirio de  varias plataformas potentes, permitiendo una inducción a ellas, y a sus procesos, bastante abrumadora en un comienzo. Sin embargo, los resultados se consideran positivos y el aprendizaje logrado.

\section*{Agradecimientos}
Agradezco a mi madre que soportó que hablaba del problema, mientras veiamos  'El Abogado del Lincoln'. También agradecer a mi amiga Camila García por sus correcciones ortográficas y correcciones varias. Y en especial, agrader a Marcela Molina por sus correcciones y apoyo.

\end{document}