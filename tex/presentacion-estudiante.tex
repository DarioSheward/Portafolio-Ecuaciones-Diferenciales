 \documentclass[../portafolio.tex]{subfiles}

% Solo agregue paquetes en el preámbulo de ../portafolio.tex

\begin{document}

\chapter*{Información personal y académica}
\addcontentsline{toc}{chapter}{Información personal y académica}
\markboth{Información personal y académica}{Información personal y académica}


%%%%%%%%%%%%%%%%%%%%%%%%%%%%%%%%%%%%%%%%%%%%%%%%%%%%%%%%%%%%%%%%%%%%%%
% Llene todos los campos, respetando tildes, mayúsculas y minúsculas.
\section*{Datos personales}

\begin{description}
\item[{Nombre completo}] Darío Alfonso Sheward Castillo % nombres y apellidos completos.
\item[{Matrícula}] 2023444987               % matrícula udec
\item[{Fecha de Nacimiento}] 7 de enero de 2005 % día de mes de año
\item[{Nacionalidad}] Chilena
\item[{E-Mail institucional}] \href{mailto:dsheward2023@udec.cl}{dsheward2023@udec.cl}
\end{description}


%%%%%%%%%%%%%%%%%%%%%%%%%%%%%%%%%%%%%%%%%%%%%%%%%%%%%%%%%%%%%%%%%%%%%%
%\section*{Breve biografía académica}
% Redacte una breve biografía (5 a 7 líneas) que incluya los
% siguientes aspectos:
% - Su nombre completo y el año en el que ingresó a la Universidad de
% Concepción.
% - Mencione su carrera actual y en qué año académico se encuentra.
% - Describa brevemente su trayectoria educativa previa a la universidad
% (por ejemplo, dónde cursó la educación media y cualquier logro académico
% relevante).
% - Mencione sus metas académicas y profesionales al finalizar el
% pregrado. ¿Qué le gustaría lograr al terminar la carrera? ¿En qué
% áreas le gustaría especializarse o trabajar?
% - Si lo considera pertinente, puede mencionar cualquier actividad
% extracurricular que haya contribuido a su formación (cursos,
% proyectos, trabajos, etc.).

Soy Darío Sheward, estudiante de segundo año de la carrera de Ciencias Físicas desde 2023. Cursé educación básica y media en el Instituto de Humanidades Alfredo Silva Santiago de Concepción. Las matemáticas siempre fueron un espacio de recreación hasta encontrarles una aplicación en fenómenos físicos. Esto me llevó a interesarme no solo en las ciencias, sino también, en la importancia del aprender a estudiar y de la persona enseñando. Del pregrado espero recolectar las herramientas necesarias para enfrentar desafíos con rigor, pensamiento crítico y resiliencia. Tras este espero poder seguir formándome como investigador. Teniendo en consideración que la labor académica conlleva formar a las próximas generaciones me parece necesario estudiar herramientas pedagógicas con el afán de aportar no solo hoy, sino también mañana al conocimiento humano. Con esta intención he participado en instancias con el equipo de difusión de la Facultad de Ciencias Físicas y Matemáticas, como también del Instituto Milenio de Investigación en Óptica.\\

%%%%%%%%%%%%%%%%%%%%%%%%%%%%%%%%%%%%%%%%%%%%%%%%%%%%%%%%%%%%%%%%%%%%%%
%\section*{Visión general e interés sobre la asignatura}
% En esta sección, reflexione y describa:
% - ¿Cuál es su percepción inicial sobre la asignatura de Física
% Computacional II? ¿Cómo se relaciona con su formación académica y sus
% intereses?
% - ¿Qué habilidades o conocimientos espera desarrollar en esta
% asignatura, específicamente en el uso de herramientas computacionales
% aplicadas a la física?
% - ¿De qué manera cree que lo aprendido en esta asignatura contribuirá a
% su desempeño en otros cursos o en su carrera profesional a futuro?
% - Si tiene alguna expectativa específica o tema de interés particular
% dentro de la asignatura, menciónelo aquí.

No me gustaría pasar la vida graficando a mano, especialmente con la cantidad datos con los que se suelen trabajar es necesario el llevarse bien con las computadoras. Espero de este curso poder fortalecer el manejo de la programación como herramienta y poder agudizar la capacidad de adaptar problemas para simularlos con métodos numéricos. El curso se presenta como una gran oportunidad de realizar un trabajo personal, tanto por su metodología clase a clase como evaluativa. Además, es de esperarse que la asignatura aporte al trabajo futuro tanto con simulaciones como plataformas de repositorios. Lo último, ha sido de un interés personal por años, siendo un entorno al que no he podido acercarme. \\

%%%%%%%%%%%%%%%%%%%%%%%%%%%%%%%%%%%%%%%%%%%%%%%%%%%%%%%%%%%%%%%%%%%%%%
%\section*{Resultados esperados de este portafolio}
% En esta sección, reflexione sobre los resultados que espera obtener al
% realizar este portafolio. Puede incluir lo siguiente:
% - ¿Qué habilidades y conocimientos espera haber consolidado al completar
% este portafolio?
% - ¿Cómo cree que el portafolio le ayudará a organizar, analizar y
% aplicar los conceptos aprendidos durante la asignatura?
% - ¿De qué manera considera que este portafolio puede servirle como
% referencia o herramienta para su futura formación académica o
% profesional?
% - Reflexione sobre cómo el proceso de autoevaluación y la inclusión de
% evidencias le permitirá comprender mejor su propio progreso.
La programación es una herramienta con un gran potencial en las manos de la persona correcta. Hasta ahora, la carrera no presenta tantas oportunidades para formarse o practicar tanta formación, por lo que espero con ansias el poder equivocarme y revisar código por una cantidad de horas moderadas. La realización de un portafolio es un desafío que no he desarrollado antes, sin embargo, me parece una gran oportunidad de lograr constancia y llevar registro de los aprendizajes. Espero, al final, poder revisar si se logró llevar disciplina durante el trabajo.
\end{document}
