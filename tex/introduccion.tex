\documentclass[../portafolio.tex]{subfiles}

\begin{document}

% No edite este archivo

\chapter*{Introducción}
\addcontentsline{toc}{chapter}{Introducción}
\markboth{Introducción}{Introducción}

Este portafolio incluirá evidencias de aprendizaje relacionadas
a la asignatura de Física Computacional II (510240), dictada en el
segundo semestre de 2024 en el departamento de Física de la Universidad de Concepción.

\medskip

Esta asignatura se enfoca en la resolución de problemas en Física
usando métodos numéricos y el lenguaje de programación
\texttt{python}. Al finalizar este portafolio, se espera que los y las
estudiantes logren:
\begin{enumerate}
\item Aplicar herramientas computacionales en la resolución
  numérica de problemas en Física.
\item Generar programas computacionales basados en algoritmos y
  conceptos de la física matemática y estadística.
\item Diferenciar, integrar y resolver ecuaciones diferenciales
  ordinarias, en forma numérica.
\end{enumerate}

\medskip

La evaluación se realizará a través de este portafolio, el cual debe
ser \textbf{entregado mediante un repositorio} de
\href{https://github.com}{GitHub} que será alojado en la organización
\href{https://github.com/fiscomp2-UdeC2024}{fiscomp2-UdeC2024}
especialmente habilitado para esta asignatura.

\begin{enumerate}
\item El repositorio será revisado periódicamente por el profesor o los ayudantes de
  la asignatura, para poder dar retroalimentación oportuna antes de la
  entrega oficial el día \textbf{viernes 29 de noviembre de 2024}. Si el protafolio no
  tiene avances en cada revisión parcial, a ocurrir a mediados de cada mes, puede costar puntaje de la nota final.

\item Las evidencias de aprendizaje serán la resolución de problemas
  \textbf{seleccionados por el estudiante} a partir de las guías del
  curso. Cada guía cuenta con una serie de problemas con dificultades
  ponderadas de 0 a 6 puntos, siendo 0 un problema de solución
  trivial o que no requiere mayor trabajo, y 6 un problema que podría ser desafiante. \textbf{Por cada guía,
  el estudiante debe elegir y responder problemas hasta completar 6
  puntos}.
\item El portafolio debe contener una sección de conclusiones final donde
  el/la estudiante resumirá los principales logros de este portafolio
  y autoevaluará su desempeño a través de una reflexión final.
\end{enumerate}

\subsection*{Criterios de evaluación}
Como criterios de evaluación, se considerará:
\begin{enumerate}
\item Coherencia de las evidencias con los resultados de aprendizaje del curso y la autoevaluación al final del documento.
\item Redacción y competencias comunicativas, incluyendo ortografía,
  gramática, sintaxis, presentación de figuras, uso de
  \LaTeX\ y explicación clara de la parte relevante de scripts de
  \texttt{python}.
\item Presentación: Claridad, limpieza y orden del documento.
\end{enumerate}

\medskip

Este portafolio, además de ser una herramienta de evaluación,
es una oportunidad para que el/la estudiante reflexione
sobre su proceso de aprendizaje, consolide sus conocimientos y
desarrolle habilidades clave en el ámbito de la computación aplicada a
la física. Se espera que este documento sirva como un registro
tangible de su progreso y de las competencias
adquiridas en la asignatura, las cuales podrán ser de gran
utilidad en futuros desafíos académicos y profesionales.

\end{document}