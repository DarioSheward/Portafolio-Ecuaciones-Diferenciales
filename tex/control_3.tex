\documentclass[../portafolio.tex]{subfiles}

\begin{document}

\chapter{Control Integrales}
\label{g0_c3}
\hfill \textbf{Fecha de la actividad:} 12 de noviembre de 2024

\medskip

Para este capítulo se nos entrega una regla de integración con pesos a determinar, en esta ocasión, daremos valor a estos últimos para que la regla sea exacta para los polinomios de hasta grado 2. Luego, se aplica la regla en una función para la que no es exacta y se comparará la solución numérica con la analítica.

\section*{Objetivos}
\begin{itemize}
\item Determinar el valor de los pesos $w_i$ de una regla de integración, partiendo de asumir para que tipo de funciones es exacta.
\item Comparar resultados numéricos y analíticos.
\end{itemize}



\section{Regla de integración entregada}

A partir de la siguiente regla de integración \eqref{g0_c3:regla} deberemos determinar sus pesos $w_i$, asumiendo que es exacta para polinomios de hasta grado 2.

\begin{equation}\label{g0_c3:regla}
\int_{-1}^{1} f(x) sin(\pi x/2) \quad dx =w_{-1} f(-1) + w_0 f(0) + w_1 f(1)
\end{equation}

Al ser 3 los pesos a determinar y 3 los tipos de polinomios para los que \eqref{g0_c3:regla} es exacto, evaluamos $f(x)=1,\quad f(x)=x \quad y \quad f(x)=x^2$.

\[
\begin{cases}
\int_{-1}^{1}sin(\pi x/2)\quad dx =1w_{-1}+1w_1  + 1w_1 \\
\int_{-1}^{1}x sin(\pi x/2)\quad dx =(-1)w_{-1}+0w_1  + 1w_1 \\
\int_{-1}^{1}x^2 sin(\pi x/2)\quad dx =1w_{-1}+0w_1  + 1w_1 
\end{cases}
\]

Ahora para poder resolver nuestro sistema de ecuaciones debemos resolver las integrales a la parte izquierda de las igualdades.

\begin{align*}
\int_{-1}^{1}sin(\pi x/2)\quad dx=& -\frac{2}{\pi}cos(\pi x/2)\bigg|_{x=-1}^{1}\\
\int_{-1}^{1}sin(\pi x/2)\quad dx=& 0
\end{align*}

\begin{align*}
\int_{-1}^{1}x sin(\pi x/2) \quad dx =&-\frac{2x}{\pi}cos (\pi x/2)+ \frac{4}{\pi^2}sin(\pi x/2)\bigg|_{x=-1}^{1}\\
\int_{-1}^{1}x sin(\pi x/2) \quad dx =&\frac{8}{\pi^2}
\end{align*}
\begin{align*}
\int_{-1}^{1}x^2 sin(\pi x/2) \quad dx =& -\frac{2x^2}{\pi} cos(\pi x/2) +\frac{4}{\pi^2}x sin(\pi x/2)\bigg|_{x=-1}^{1}\\
\int_{-1}^{1}x^2 sin(\pi x/2) \quad dx =& 0
\end{align*}

El sistema queda como:
\[
\begin{cases}
0 \quad=w_{-1}+w_1  + w_1 \\
8/\pi^2=-w_{-1}+ w_1 \\
0\quad =w_{-1} + w_1 
\end{cases}
\]
Determinamos; $w_{-1}=\pm 4/\pi^2$, $w_{1}=\mp4/\pi^2$ y $w_{0}=0$. Así la regla queda descrita en \eqref{g0_c3:resultado}

\begin{equation}\label{g0_c3:resultado}
\int_{-1}^{1} f(x) sin(\pi x/2) \quad dx =\frac{4}{\pi^2} f(-1) -  \frac{4}{\pi^2} f(1)
\end{equation}

\subsection{Comparación con resultados analíticos}

Se nos entrega el siguiente resultado analítico \eqref{g0_c3:analitic}. A continuación aplicaremos la regla \eqref{g0_c3:regla} en la integral 

\begin{equation}\label{g0_c3:analitic}
\int_{-1}^{1} x^2sin^2(\pi x/2) \quad dx =\frac{1}{3}+ \frac{2}{\pi^2} \simeq 0.53598.
\end{equation}

Consideramos $f(x)=x^2sin(\pi x/2)$ y calculamos 

\begin{align*}
\int_{-1}^{1} x^2sin^2(\pi x/2) \quad dx =\frac{4}{\pi^2}(1)  -  \frac{4}{\pi^2}(-1) \\
\int_{-1}^{1} x^2sin^2(\pi x/2) \quad dx =\frac{8}{\pi^2}\simeq 0.81056.
\end{align*}

Entonces el error absoluto entre los cálculos es 

\begin{equation}
E_{abs}=|0.81056-0.53598|=0.27458
\end{equation}
y el error relativo 
\begin{equation}
E_{rel}=\frac{0.53598}{0.27458}=51.22\%
\end{equation}
\subsection{Error del método}
Sabemos que la regla es exacta para polinomios de hasta grado 2. En general, los errores de las cuadraturas gaussianas, como esta, son proporcionales a las derivadas superiores del integrando. Al evaluar una función compuesta por un polinomio y por la función seno, las derivadas no se anulan como para los casos en los que es exacta. Al tratarse de polinomios de hasta grado 2, la condición para su exactitud debe depender de un factor $f'''(\xi)$, sea $\xi$ un número entre los límites de integración, ya que $f^{(p)} (\xi)=0$ mientras p>2.
%Al tratarse de una cuadratura gaussiana, agregamos la función de pesos en la integral, sea en este caso $w(x)=1$
%\begin{equation}
%I(x_i,x_{i+1}) = \int_{-1}^{1} f(x) w(x) sin(\pi x/2) \quad dx 
%\end{equation}


\section{Análisis de resultados}
La determinación de los pesos permite encontrar una regla con limitaciones, sin embargo, útil para casos específicos. La regla no resultó óptima para la función en la que se aplicó, entregando un porcentaje de error relativo, en general, insatisfactorio. Se sugiere, redefinir la regla, determinar los pesos de \eqref{g0_c3:regla} pero para $\int_{-1}^{1} f(x) sin^2(\pi x/2) \quad dx$ y considerar $f(x)=x^2$. O alguna otra regla de integración distinta a una cuadratura gaussiana.
\section*{Conclusiones}
Las capacidades de los métodos de integración numérica es uno de los principales sujetos de aprendizaje. La implementación de series de Taylor y sistemas de ecuaciones para caracterizar a la regla resulta un gran ejercicio a reiterar a futuro, si se trabaja con integración numérica para ciertos tipos de funciones. A partir de esto se consideran a los objetivos planteados logrados.

\end{document}
