\documentclass[../portafolio.tex]{subfiles}

\begin{document}

\chapter{Puntos de equilibrio de un péndulo invertido}
\label{g5_ej8}
\hfill \textbf{Fecha de la actividad:} 30 de noviembre de 2024

\medskip

En este capítulo se utilizan métodos de derivación numérica y búsqueda de ceros para determinar los puntos de equilibrio, y sus estabilidades, para un sistema de un péndulo invertido conectado a un resorte. Se revisaron estos puntos para distintos coeficientes de amortiguamiento que presenta el sistema. Luego, se realiza un breve análisis de las implicancias físicas de los resultados.

\section*{Objetivos}
\begin{itemize}
\item Determinar los puntos de equilibrio de la ecuación normalizada de un péndulo invertido conectado a un resorte, numéricamente.
\item Determinar la estabilidad de los puntos de equilibrio, numéricamente.
\end{itemize}
\medskip
Nos ponemos en el caso de un péndulo invertido conectado a un resorte. El ejercicio normaliza la ecuación de energía potencial de tal manera que:
\begin{equation}\label{g5_ej8:eq_dif}
u(\theta)=cos \theta + \beta sin^2\theta
\end{equation}
Donde $\beta)kd^@/2mgl$ y $u=U/mgl$, con $U$ como energía potencial, $k$ la constante del resorte, $d$ la altura de resorte, $g$ la acelaración de gravedad, mientras, $m$ y $l$ son la masa y largo del péndulo.
\begin{figure}
\centering
\includegraphics[scale=0.6]{../img/g5_imag_ref.png}
\caption{Péndulo invertido conectado a un resorte de constante k.} \label{g5_ej8_graf_ref}
\end{figure}
Trabajaremos en el intervalo de $0\leq \theta \leq \pi /2$ y para $0\leq \beta \leq 1$ usaremos 11 números equiespaciados en el intervalo. Consideremos a $\beta$ como un coeficiente de amortiguamiento logrado por el resorte.
\section{Puntos de equilibrio}
Para determinar los puntos de equilibrio de la energía potencial tenemos que igualar la derivada de ésta a cero, $u'(\theta)=0$. Realizaremos este proceso numéricamente. A través de un script de Python seleccionaremos cierta cantidad de datos en el intervalo de $\theta$ realizaremos un proceso de derivación centrada, explicado en capítulo \ref{g2_ej9}, sobre el intervalo evaluado en \eqref{g5_ej8:eq_dif}.Esto se repite para cada valor de $\beta$ que consideremos. Este proceso será explicado a detalle a continuación.
\begin{minted}{python}
def energia(theta,beta=1):	#Se define la función de energía potencial donde evaluar el ángulo theta.
    return np.cos(theta)+beta*(np.sin(theta)**2)
theta=np.linspace(0,0.5*np.pi,500)	#Se define el intervalo de ángulos a considerar.
betas=np.linspace(0,1,11)	#Se determinan los betas a trabajar.
for beta in betas:		#Se considera un valor de beta a la vez a través del ciclo.
    beta=round(beta,1)	#Se redondea el valor de beta al décimal para omitir error de redondeo al graficar.
    fx=energia(theta,beta)	#Se evalúa en la ecuación de energía potencial.
    xs, varEnergia =derivadacentrada(fx, theta, 1)	#Se derivan los resultados de la evaluación anterior.
    plt.plot(xs , varEnergia, label=r'$\beta =$'+f'{beta}')	#Se grafica, para cada valor de beta, la variación de energía potencial respecto al ángulo theta.
\end{minted}
El gráfico mostrando la variación de energía potencial respecto al ángulo es la figura \ref{g5_ej8_graf0}.

\begin{figure}
\centering
\includegraphics[scale=0.75]{../img/ej_8_0.png}
\caption{Gráfico variación de energía potencial respecto al ángulo.} \label{g5_ej8_graf0}
\end{figure}
Ahora para determinar que ángulos la variación de energía potencial es cero utilizaremos el método de la bisección. Sea este un método reiterativo basado en el teorema del valor intermedio \citep{Stewart2001}, acotando el intervalo de búsqueda según los signos entre las imágenes de los extremos. Sea el método explicado a detalle a continuación en el script de Python. Notar que el método fue programado para revisar subintervalos\footnote{Esto se debió a que el problema se inició considerando el intervalo de $\theta \in [0,2\pi]$, sonde existen múltiples ceros para las derivadas calculadas.}
\begin{minted}{python}
#A continuación se revisará como se programó el método de la bisección.
def biseccion(xs, fxs, tol=1e-3):	#El método requiere un arreglo de datos donde buscar y las imagenes de este arreglo evaluado en una función.
    ceros =[]	#Este arreglo se utiliza para guardar lo ceros encontrados.
    franja=15	#Sea "franja" el tamaño del subintervalo con el que se da inicio a la búsqueda.
    i =0
    while i <= len(xs)-franja:	#Sea el ciclo controlado por un contador que se detinee al haber pasado por todos los ángulos theta.
        x =xs[i:i+franja]	#Se seleccionan las preimágenes a utilizar a partir del tamaño de #franja#.
        fx =fxs[i:i+franja]	#Se hace lo mismo para las imágenes.
        a, b =x[0], x[-1]	#Se determinan los extremos de los intervalos con el que trabaja el método.
        fa, fb =fx[0], fx[-1]	#Se definen las imágenes de los extremos, sean estos los que comparará el método.
        if fa*fb>0:	#El método compara las imágenes, si su múltiplicación es postiva, se deduce que no hay un cero en el intevalo...
            i += franja	#Por lo que se pasa al siguiente subintervalo.
            continue
        while np.abs(b-a)>tol: 	#De tener los extremos de los intervalos imágenes de distinto signo, queriendo decir que hay un cero en el subintervalo, y la diferencia entre los  extremos sea menor a una tolerancia definida:
            c =(a+b)*0.5		#Se evalúa el punto medio del intervalo.
            dif =np.abs(x-c)		#En este caso, al realizar el proceso númericamente, se busca el ángulo conocido más cercano al punto medio.
            ic=np.argmin(dif)	#Se considera el índice del ángulo más cercano para luego seleccionar su imagen y compararla con las imágenes extremas.
            fc=fx[ic]	#Sea ésta su imagen.
            if fa*fc<0:	#Si la múltiplicación entre las imágenes del extremo menor del intervalo y el punto medio es negativo, entonces el cero está entre estos, por lo que el punto medio es el nuevo extremo superior del intervalo.
                b=c
                fb=fc
            else:		#De lo contrario el punto medio pasa a ser el extremo inferior del intervalo donde se trabaja.
                a=c
                fa=fc
        ceros.append((a+b)*0.5)	#Cuando se llegue a un subintervalo de tamaño menor a la tolerancia se guardará el punto medio del último intervalo resgitrado.
        i+=franja	#Luego se reitera el proceso en el siguiente subintervalo.
    return ceros		#Se retornan todos los ceros encontrados en el intervalo entregado.
\end{minted}
Entonces al determinar los ceros los podemos graficar sobre las variacones de energía potencial. Notar que estos puntos de equilibrio solo ocurren para $\beta>0.5$. Por esto solo se grafican las curvas con ceros no triviales en la figura \ref{g5_ej8_graf1}

\begin{figure}
\centering
\includegraphics[scale=0.75]{../img/ej_8_1.png}
\caption{Graficado de ceros en la variación de energía potencial respecto al ángulo.} \label{g5_ej8_graf1}
\end{figure}

Ahora conociendo los puntos de equilibrio no triviales, sean los triaviales aquellos $u'(0)=0$, los graficamos respecto al $\beta$ con el cual ocurren. Sea la figura \ref{g5_ej8_graf2}. Acá ya podemos notar cierta relación entre $\beta$ y el punto de equilibrio.

\begin{figure}
\centering
\includegraphics[scale=0.75]{../img/ej_8_2.png}
\caption{Ángulos de equilibrio no triviales para cada $\beta$.} \label{g5_ej8_graf2}
\end{figure}

\section{Estabilidad de los puntos}
Para determinar la estabilidad de los puntos no triviales encontrados los graficaremos respecto a la segunda derivada de la energía potencial del sistema \citep{chasnov_diferenciales_2024}. 
El criterio de la segunda derivada establece una relación entre un punto de equilibrio y si es un, máximo o mínimo, es decir, describe su estabilidad. En este caso, sea $\theta^*$ un punto de equilibrio: 
\begin{itemize}
\item Si $U''(\theta^*)<0$, entonces $\theta^*$ es un punto estable. 
\item Si $U''(\theta^*)>0$, entonces $\theta^*$ es un punto inestable.
\item Si $U''(\theta^*)=0$, entonces $\theta^*$ es un punto marginalmente estable.
\end{itemize}
Para determinar $U''(\theta^*)$ se derivó numéricamente la energía potencial y se replicaron los procesos pasados para relacionar una preimagen con una imagen de esta segunda derivada.
Luego se graficaron los puntos de estabilidad triviales junto a los no triviales, siendo el resultado de esto la figura \ref{g5_ej8_graf3}.

Por el criterio de la segunda derivada determinamos que todos los puntos de equilibrio no triviales son puntos estables, mientras que los puntos triviales son estables para $\beta <0.5$, e inestables para $\beta > 0.5$. En el caso del punto de estabilidad trivial para $\beta =0.5$ se trata de un punto marginalmente estable.
\begin{figure}
\centering
\includegraphics[scale=0.75]{../img/ej_8_3.png}
\caption{Ángulos de equilibrio respecto a la segunda derivada de enegía potencial, por cada $\beta$.} \label{g5_ej8_graf3}
\end{figure}
Cada uno de estos comportamientos se puede visualizar a través del gráfico de fases representado en la figura \ref{g5_ej8_graf4}.
\begin{figure}
\centering
\includegraphics[scale=0.75]{../img/ej_8_4.png}
\caption{Gráfico de fases: Comparación de comportamientos para distintos $\beta$.} \label{g5_ej8_graf4}
\end{figure}
\section{Análisis de resultados}
El análisis de puntos de equilibrio y sus características resulta un fenómeno bien documentado el cual es comparable con múltiples fuentes. Los resultados obtenidos coinciden con lo esperado, aunque los casos más simples suelen ser los menos populares entre las referencias. 
En cuanto al fenómeno físico se puede notar que, en cuanto a información del sistema físico \ref{g5_ej8_graf_ref}, mientras menor sea $\beta$ menor variación de energía potencial presenta el sistema. También, para $\beta>0.5$ se presentan puntos de equilibrio inestables, causando comportamientos inusuales en un péndulo común. Con presencia de variaciones en trayectoria y velocidades.

Los scripts resultaron ser sumamente específicos y se generan líos de índices, por lo que para la divulgación de los métodos se estimaría conveniente mejorar la redacción del código. En cuanto a los resultados numéricos se reitera, todo coincide con lo presupuestado por las referencias.
\section*{Conclusiones}
Los resultados obtenidos sobre el sistema del péndulo invertido conectado a un resorte fueron obtenidos en su totalidad con métodos numéricos. El uso de métodos de derivación numérica y búsqueda de ceros hicieron de este problema más simple de abordar, con cotidianos desafíos en cuanto al manejo de arreglos en Python. La determinación de los puntos de equilibrio y sus características ofrecen mucha información sobre el sistema físico estudiado, sin ambargo, se sugiere revisar simulaciones dinámicas para facilitar la percepción del fenómeno. Se estiman logrados los objetivos propuestos al comienzo de la actividad.
\section*{Agradecimientos}
A Rigoberto McPato por escucharme y a su nuevo compañerito, aún sin nombre.
\end{document}
