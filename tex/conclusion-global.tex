\documentclass[../portafolio.tex]{subfiles}

\begin{document}

\chapter*{Conclusiones}
\addcontentsline{toc}{chapter}{Conclusiones}
\markboth{Conclusiones}{Conclusiones}

\hfill \textbf{Fecha de presentación:} Miércoles 4 de diciembre de 2024

\medskip

% ESTE CAPÍTULO LO DEBES LLENAR AL FINALIZAR LA ASIGNATURA, por supuesto antes de la fecha de entrega.

%--------------------------------------------------------------------------------
% Inicie con un resumen breve de cuáles eran los objetivos del portafolio;
Este portafolio se inició con la intención de poder desarrollar habilidades tanto de programación como de resolución de problemáticas físicas. Esperando que al concluir el curso se dispongan de nuevas herramientas numéricas para afrontar las situaciones.
%--------------------------------------------------------------------------------
% [Resumen de los contenidos]
% - Un resumen MUY breve de cuáles son las evidencias de aprendizaje que incluyó en este portafolio. Algo como "En el capítulo 1, se derivó numéricamente la función coseno, usando un esquema de derivadas centradas, para estudiar el error absoluto con respecto a la derivada analítica de la misma función."
% - Incluya una breve reflexión de lo que aprendió en cada actividad, lo que faltó aprender, lo que no se entendió y lo que sí se entendió bien.
% - Haga lo anterior por cada evidencia de aprendizaje.

Durante el desarrollo de toda esta actividad se pudieron revisar múltiples contenidos. En los primeros 5 capítulos se revisan los controles realizados durante el curso, tratando tanto los errores de truncamiento y redondeo de un esquema de derivación numérica, las capacidades numéricas y gráficas de Python, la determinación de los pesos de una regla de integración exacta, utilizándola en una función para la que no es exacta; la normalización de un sistema de ecuaciones diferenciales ordinarias, incluyendo su resolución numérica mediante el método del salto de la rana; hasta probar las capacidades de una generalización del método de Newton-Raphson. En los siguientes capítulos se estudian las capacidades de Python para tratar con números sumamente grandes o sumamente pequeños, usando secuencias, como los números de Catalán, y series, como la serie de Maclaurin de la función exponencial; como también se utilizó el lenguaje para facilitar el análisis de los resultados, como se hizo al estudiar la secuencia de Lucas. Después se tratan temáticas de diferenciación numérica, analizando los errores generados por distintos esquemas, determinando los pesos de un esquema de derivación de 3 nodos y estudiando métodos de mitigación de errores a la hora de derivar numéricamente. Siendo este último una de las ocasiones donde se pudieron revisar métodos ajenos a las indicaciones del profesor, aplicando el filtro de la media móvil. Tras estos, se exploran las capacidades de la integración numérica, aplicándola en la estimación de la función Gamma, como también en la resolución de una ecuación integral. Los últimos capítulos presenta el estudio de problemáticas numéricas de ecuaciones diferenciales ordinarias, desde la comparación de métodos hasta su aplicación, permitiendo el análisis de puntos de equilibrio y su estabilidad; para tratar también métodos de búsqueda de ceros, aplicándolos en sistemas físicos, como el caso del péndulo invertido, y los polinomios, como los de Legendre y de Hermite, creando un primer acercamiento a ciertas funciones de Python con las que poder comparar resultados de estas estimaciones.


Cada uno de los ejercicios revisados resaltó algún aspecto de la temática o herramienta a estudiar. En general, cada ejercicio fue seleccionado con el afán de desarrollar aspecto distinto de la labor realizado, para poder trabajar tanto aspectos analíticos, numéricos y como también físicos. Notar que en más de alguno de los ejercicios convergen estos campos, sin embargo, tienden a aparecer al final del curso. 

%--------------------------------------------------------------------------------
% [Autoevaluación del alumno/a]
% Realice una reflexión de cómo trabajó usted, qué cree haber hecho bien y mal en el curso, qué le gustaría  hacer a futuro (en la forma de estudiar y en cómo cree que aplicará los contenidos de este portafolio en el futuro), cómo han distribuido su trabajo a lo largo del trabajo en este portafolio.

En cuanto al aspecto personal del desarrollo del portafolio se experimentaron múltiples dificultades. Empezando por un desapego al formato y a la metodología que tuvo que ser corregida en varias ocasiones. A la vez, hubo problemas técnicos en demasía durante el inicio del trabajo, lo cual persiguió a buena parte de los estudiantes por lo menos un par de semanas.


En buena parte de los ejercicios se presenta la posibilidad de analizar más allá de lo que pide el enunciado, esto, usualmente, daba la oportunidad de explorar métodos ajenos al curso, sin embargo, esto generaba mayor carga de trabajo, por lo que dificultaba el desarrollo del portafolio en general. Esto fue una dificultad completamente personal, ya que no había necesidad de sobrecargarse de tal manera.
Aun así, la resolución de los problemas puede tomar el tiempo más tiempo del deseado, es acá donde el conversar con los compañeros de este curso pudo resultar útil, sin embargo, no se pudo hacer tanto debido a la carga académica. El último factor pudo influir menos de darse los puntajes de los ejercicios con anterioridad.


El plazo de entrega de este portafolio fue un potencial problema (nada que unas noches de poco sueño no resuelvan). Ante el cambio de plazo realizado y el cambio en las condiciones de entrega, se decidió realizar la entrega en el plazo extendido con la intención de poder garantizar la entrega de un trabajo integral.
Respecto a la calidad del trabajo, prefiero dudar de ella, antes de ir confiado de que es ideal. Espero poder contar con retroalimentación de este y que sea de apoyo para poder mejorar este estilo de trabajo aquí en adelante.

% --------------------------------------------------------------------------------
% [Evaluación del curso]
% - Realice una comparación entre sus expectativas iniciales, tal como las describió en la sección de presentación, y lo que realmente aprendió y experimentó a lo largo del curso. ¿Se cumplieron sus expectativas sobre la asignatura? ¿En qué medida?
% - ¿Qué aspectos del curso o del portafolio superaron, igualaron o no alcanzaron sus expectativas iniciales?
% - Agregue sugerencias para futuras versiones del curso, para que estudiantes de generaciones venideras se beneficien de una aplicación mejorada de este instrumento de evaluación.
% - ¿Cuál es la evidencia de este portafolio que usted cree es mejor/más relevante/en la que aprendió mejor? ¿Qué diferencia a esa evidencia del resto incluido en este portafolio?
% - ¿Puede evaluar la utilidad de este portafolio?

El curso se presentó en un comienzo como un curso de física donde se aplicarían métodos numéricos para la resolución de problemáticas de fenómenos físicos. Tras las primeras semanas del curso, me pareció extraña la gran cantidad de programación que vimos sin presencia de sistemas físicos directamente.  Sin embargo, con el pasar de las semanas se entendió la necesidad de repasar las capacidades computacionales de nuestras herramientas, como también repasar nuestras competencias en el uso de las últimas. Al final del curso, se considera que el curso presenta una apreciada progresividad que resulta sumamente agradable. Durante estas 16 semanas todo tópico revisado resultó útil para el desarrollo tanto personal como académico de mí. Me parece que el curso es una gran introducción a los métodos numéricos más utilizados, sin embargo, considero que todos los aprendizajes reflejados en el portafolio, para maximizar su utilidad, han de ser complementados con revisiones de nuevos métodos y la profundización de los ya conocidos. 


No puedo terminar sin mencionar lo agradecido que estoy del curso al por llevarme a aprender GitHub, explotar capacidades de Visual Studio Code, TeXMaker, Python y \LaTeX , profundizar en conceptos de cálculo y álgebra, obligarme a recordar como se trabajan las ecuaciones diferenciales ordinarias, hacer que vuelva a aprender a resolver estas últimas y recordame la importancia de los ceros en sistemas físicos. A partir de todo lo anterior puedo decir que el desarrollo del portafolio de este curso resultó sumamente productivo, en cuanto a aprendizaje se refiere, mientras que en cuanto al tiempo que se consume en su desarrollo, se considera mayor al estimado por el programa del curso.
\end{document}